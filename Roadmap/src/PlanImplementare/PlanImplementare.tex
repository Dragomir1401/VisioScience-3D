\documentclass[a4paper,12pt]{article}
\usepackage[utf8]{inputenc}
\usepackage{geometry}
\geometry{a4paper, margin=1in}
\title{VisioScience3D - Platforma Web pentru Invatare prin Vizualizari 3D}
\author{Andrei Dragomir}
\date{Februarie 2025}

\begin{document}

\maketitle

\section*{Prezentare generala a proiectului}
Proiectul propune dezvoltarea unei platforme educationale, denumita \textbf{VisioScience3D}, 
dedicata vizualizarii interactive 3D si simularilor pentru discipline de nivelul gimaziu/liceu
precum Matematica, Fizica, Chimie si altele pentru timpul liber ca si Astronomia.

Platforma isi propune sa vina in ajutorul profesorilor si elevilor pentru a face invatarea mai 
rapida si usoara prin vizualizari interactive si simulari ale fenomenelor stiintifice. De multe orice
profesorii sunt nevoiti sa explice si sa deseneze pe table concepte 3D complexe care sunt greu de
inteles pentru elevi. Aceasta platforma Web va functiona ca un loc centralizat pentru concepte 
3D si fenomene pe care copiii le invata in scoala si liceu.

\section*{Tehnologii utilizate}

Platforma va avea la baza un backend scris in Go pentru gestionarea datelor si interactiunea cu
utilizatorii, iar frontend-ul va fi realizat cu React, Tailwind CSS si Three.js pentru randarea graficii 3D.
Simularile fizice vor fi implementate cu ajutorul Cannon.js. Ambele framework-uri folosesc la 
baza WebGL pentru randarea graficii 3D.

\section*{Structura si continutul}
Platforma va include un meniu cu disciplinele stiintifice si subiecte specifice fiecarei materii:
\begin{itemize}
    \item \textbf{Landing page:} Meniul principal va fi construit folosind elemente 3D pentru alegerea materiilor
    si va avea o experienta de navigare 3D pentru selectare.
    \item \textbf{Matematica:} Vizualizari interactive ale formelor geometrice 
    (ex: cercuri cu raza si diametru ajustabile), graficul functiilor si formule asociate.
    Acestea vor ajuta elevii sa vizualizeze invatarea geometriei printr-un mod interactiv si vizual.
    \item \textbf{Fizica:} Simulari de probleme clasice precum planuri inclinate cu vectori
    ai fortelor si unghiuri ajustabile, miscarea proiectilului in aruncari veticale si eliptice si altele.
    Elevii vor putea vizualiza fortele cum actioneaza asupra obiectelor in mod interactiv. 
    Astfel se pot intelege mai usor distributiile de energie cinetica si potentiala in timpul miscarii
    si fortele care actioneaza pentru a mentine obiectele in echilibru sau in miscare. Vor exista
    si simulari pentru tipuri de miscari ca cea rectilinie, rectilinie uniforma, ondulatorie si miscarea
    pendulurilor.
    \item \textbf{Chimie:} Modele 3D ale moleculelor compusilor de baza pentru a vedea unghiurile
    la care se formeaza legaturile si structura tridimensionala a moleculelor de baza invatate in gimaziu
    si liceu.
    \item \textbf{Astronomie:} Reprezentari 3D ale planetelor, orbitelor si satelitilor acestora,
    pentru vizualizarea minimalista a caracteristicilor de baza ale sistemului solar si a fenomenelor
    astronomice.
\end{itemize}

\section*{Interactiunea cu utilizatorii}
Utilizatorii vor putea interactiona cu obiectele 3D prin miscarea mouse-ului, ajustarea parametrilor
si vizualizarea formulelor corespunzatoare. 
Platforma va avea la baza un meniu de alegere a disciplinelor, fiecare avand o selectie de elemente
de vizualizare sub forma de capitole ca pagini separate unde elementele principale vor fi fereastra
de vizualizare 3D si un panou lateral cu setarile si formulele asociate.
Navigarea va fi intuitiva si accesibila pentru elevi de orice varsta cu cunostinte de la nivel scazut
pana la niveluri mai avansate cand se vizeaza forme sau fenomene mai complexe.

\section*{Extensii viitoare}
Extensiile viitoare pot include:
\begin{itemize}
    \item adaugarea de noi discipline
    \item implementarea de teste interactive pentru fiecare disciplina
    \item implementarea de sesiuni de autentificare pentru salvarea progresului
    \item adaugarea de roluri de profesor pentru compilarea de lectii personalizate pentru clase de elevii
    \item crearea de leaderboard-uri sau provocari pentru a stimula competitia intre elevi prin efectul de gamificare.
\end{itemize}

\section*{Concluzie}
VisioScience3D isi propune sa ofere o experienta de invatare captivanta prin combinarea 
graficii interactive cu principiile fundamentale ale stiintei, facand educatia mai accesibila si mai atractiva.

\end{document}
