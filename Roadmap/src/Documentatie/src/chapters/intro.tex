\chapter{Introducere}
\label{chapter:intro}

% \textbf{This is just a demo file. It should not be used as a sample for a thesis.}\\
% \todo{Remove this line (this is a TODO)}

\section{Context}
\label{sec:proj}
% spatiu intre liinii
\onehalfspacing
Educația este un domeniu de bază al socitetății, iar tehnologia joacă un rol din ce în ce mai important 
în acest sector. În special, într-o lume în care platformele sociale si mediul de interacțiune video subsection
de bază pentru tineri, este esențial să se dezvolte soluții educaționale care să fie atractive și eficiente în 
procesul clasic de învățare.

\subsection{Definirea problemei}
\label{sub-sec:proj-scope}
\onehalfspacing
În acest context, problema pe care o abordăm este crearea unei platforme educaționale interactive care să integreze
tehnologii moderne și simulări 3D, pentru a face din procesul de învățare o experiență captivantă și eficientă pentru
elevi de gimnaziu și liceu. Această platformă va permite accesul vizual și facil la informații complexe de matematică,
fizică, chimie, astronomie și informatică.

Studenții vor putea explora concepte și interacționa cu simulări 3D, dar și să participe la teste și evaluări
pentru a-și verifica cunoștințele. De asemenea, profesorii vor avea la dispoziție un instrument pentru a crea teste și
a gestiona clasele de elevi, facilitând astfel procesul de predare și evaluare. Platforma va avea un sistem interactiv de 
navigare, recunoaștere a rezultatelor si răsplătirea progresului prin gamificare tot prin interacțiune 3D, ceea ce va
îmbunătăți experiența utilizatorilor și va stimula învățarea activă.


\subsection{Obiective}
\label{sub-sec:proj-objectives}

\onehalfspacing
Obiectivele principale ale acestui proiect sunt:
\begin{itemize}
    \item Crearea unei platforme educaționale interactive care să integreze simulări 3D și tehnologii moderne.
    \item Dezvoltarea unui sistem de gestionare a testelor și evaluărilor pentru profesori și elevi.
    \item Implementarea unui sistem de gamificare pentru a stimula învățarea activă și implicarea utilizatorilor.
    \item Asigurarea accesibilității și ușurinței în utilizare pentru elevi și profesori.
    \item Crearea unui mediu de învățare captivant și eficient care să faciliteze înțelegerea conceptelor complexe.
    \item Integrarea unui sistem de raportare și monitorizare a progresului utilizatorilor.
    \item Crearea unei interfețe prietenoase și intuitive care să faciliteze experiența profesorilor în gestionarea
     claselor, testelor, elevilor și a rezultatelor.
\end{itemize}

% \fig[scale=0.5]{src/img/reporting-framework.pdf}{img:report-framework}{Reporting Framework}


\subsection{Susținere științifică}
\label{sub-sec:proj-scientific-support}


Multe discipline STEM implică concepte abstracte foarte dificil de vizualizat, ceea ce poate scădea interesul elevilor.
De exemplu, chimia este adesea percepută ca “prea abstractă” deoarece elevii nu pot vizualiza ușor concepte precum
structura moleculară sau reacțiile chimice. 

% Importanța vizualizărilor 3D și a tehnologiilor interactive în predarea STEM
% Multe discipline STEM implică concepte abstracte dificil de vizualizat, ceea ce poate scădea interesul elevilor. De exemplu, chimia este adesea percepută ca “prea abstractă” deoarece elevii nu pot vizualiza ușor concepte precum orbitali atomici sau structura particulelor
% stemeducationjournal.springeropen.com
% . Reprezentările vizuale (diagrame, modele 3D, animații) au fost dezvoltate tocmai pentru a facilita înțelegerea acestor date și idei complexe
% stemeducationjournal.springeropen.com
% . Tehnologiile interactive (simulări pe calculator, realitate augmentată/virtuală, laboratoare virtuale etc.) permit concretizarea fenomenelor teoretice, făcând învățarea mai tangibilă și intuitivă. Un studiu de sinteză (2010–2022) arată că integrarea VR/AR în predarea K-12 a cunoscut o creștere accelerată și aduce avantaje clare: un impact pozitiv asupra elevilor (motivație, implicare) și îmbunătățirea proceselor de predare și învățare
% link.springer.com
%  (în ciuda unor provocări precum posibile distrageri sau necesitatea formării profesorilor). La nivel pedagogic, includerea materialelor vizuale și interactive oferă multiple beneficii care susțin învățarea. Experții subliniază că elevii devin mai motivați și angajați atunci când informația le este prezentată vizual (imagini, grafice, modele) și nu doar verbal. De exemplu, un cadru didactic enumeră următoarele avantaje ale folosirii suporturilor vizuale în clasă
% revistaprofesorului.ro
% :
% Creșterea motivației și a implicării: elevii manifestă mai mult interes și participă mai activ la lecție când se folosesc imagini, scheme sau animații, în locul expunerii exclusiv orale
% revistaprofesorului.ro
% . Vizualul captează atenția și reduce efortul mental necesar înțelegerii, făcând informația mai accesibilă.
% Claritate în înțelegerea conceptelor abstracte: elementele vizuale (diagrame, simulări 3D, hărți, modele) îi ajută pe elevi să perceapă și să relaționeze mai ușor concepte complicate, realizând conexiuni rapide cu mai puțin efort
% revistaprofesorului.ro
% . Astfel se facilitează gândirea vizual-spațială – de exemplu, un model 3D al moleculelor permite observarea relațiilor spațiale pe care textul sau vorbirea nu le pot transmite la fel de eficient.
% Impact emoțional și motivațional sporit: materialele vizuale au adesea un impact afectiv puternic – imagini sugestive, animații atractive – ceea ce sensibilizează și crește interesul elevilor pentru subiectul studiat
% revistaprofesorului.ro
% . Emoția și curiozitatea stârnite de o simulare interactivă (de ex. o animație a sistemului solar) pot motiva elevii să exploreze mai mult.
% Un consens emergent în literatura educațională este că aceste instrumente vizuale și interactive pot contribui la formarea unei atitudini pozitive față de științe. Prin stimularea imaginației și oferirea de interacțiuni practice, elevii înțeleg utilitatea concretă a conceptelor teoretice și își dezvoltă o atitudine proactivă în învățare
% stemeducationjournal.springeropen.com
% stemeducationjournal.springeropen.com
% . În plus, vizualizările 3D și simulările pot sprijini personalizarea predării pentru elevii cu stiluri de învățare diferite (vizuali, auditivi, kinestezici), oferindu-le șansa să asimileze informația în modul optim pentru ei
% iteach.ro
% revistaprofesorului.ro
% . Astfel de tehnologii se aliniază cu nevoile noilor generații de nativi digitali, obișnuiți cu conținut multimedia, făcând procesul educativ mai relevant și conectat la lumea reală.
% Îmbunătățirea înțelegerii și retenției informației (față de metodele tradiționale)
% Pe lângă atractivitate, esențială este eficiența acestor metode: în ce măsură utilizarea vizualizărilor 3D și a simulărilor interactive conduce la o mai bună înțelegere și retenție a cunoștințelor comparativ cu predarea tradițională (bazată pe text sau prelegere). Numeroase studii din ultimii ani indică rezultate superioare atunci când se folosesc astfel de tehnologii educaționale inovative:
% Experiment cu elevi (Cehia, 2019): utilizarea modelelor 3D și a animațiilor în predarea științelor la 565 de elevi de gimnaziu și liceu a condus la creșterea semnificativă a motivației intrinseci pentru învățare (efect mediu Hedges’ g ~0,38 pentru interes/implicare) și la îmbunătățirea performanțelor la testele de cunoștințe, comparativ cu metodele statice
% stemeducationjournal.springeropen.com
% . Elevii din grupul care a folosit vizualizări 3D au obținut scoruri mai mari în special la chimie decât cei din grupul tradițional, iar efectele pozitive au fost cele mai puternice la elevii mai tineri (11–13 ani) și la discipline precum Chimie și Biologie (g ~0,72)
% stemeducationjournal.springeropen.com
% . Cercetătorii au concluzionat că introducerea acestor materiale vizuale dinamice este benefică în procesul didactic și au recomandat profesorilor integrarea lor constantă în predare
% stemeducationjournal.springeropen.com
% .
% Meta-analiză (2025) – predare clasică vs. metode 3D interactive: o analiză ce a inclus 18 studii controlate (1.077 participanți) a comparat lecțiile tradiționale cu instruirea ce combina modele 3D și învățarea bazată pe probleme (problem-based learning). Rezultatele agregate arată avantaje substanțiale în favoarea metodei 3D interactive: cunoștințele teoretice ale cursanților au fost cu 1,62 deviații standard mai bune (SMD = 1,62), iar abilitățile practice cu 2,29σ mai ridicate decât în cazul învățării prin prelegeri clasice
% pubmed.ncbi.nlm.nih.gov
% . De asemenea, studenții instruiți cu ajutorul vizualizărilor 3D au demonstrat o înțelegere mai aprofundată a structurilor complexe și un nivel de satisfacție mai mare față de procesul de învățare
% pubmed.ncbi.nlm.nih.gov
% . Concluzia autorilor a fost că integrarea vizualizărilor 3D cu metode interactive reprezintă un “metodă de predare foarte eficientă”, care nu doar îmbunătățește scorurile academice, dar dezvoltă și abilități precum comunicarea și colaborarea
% pubmed.ncbi.nlm.nih.gov
% .
% Simulări VR în laborator (Italia, 2024): un studiu experimental realizat cu liceeni în domeniul biotehnologiei a testat impactul realității virtuale asupra învățării practice. Elevii care au folosit simulări VR imersive în completarea orelor tradiționale au înregistrat progrese semnificative la testele de cunoștințe: scorul mediu a crescut de la 3,7 (pre-test) la 6,6 (post-test) din 10 puncte posibile
% frontiersin.org
% . Mai important, după 30 de zile, cunoștințele dobândite s-au menținut aproape la același nivel (fără scădere statistic semnificativă la retestare)
% frontiersin.org
%  – indicând o retenție îmbunătățită a informației datorită învățării interactive. În plus, elevii implicați în VR s-au declarat mai încrezători în abilitățile lor practice și mai interesați de domeniul studiat, sugerând că tehnologia imersivă le-a amplificat atât competențele, cât și motivația. (Alte cercetări confirmă că folosirea suporturilor vizuale crește retenția cu ~30–40%, permițând studenților să își amintească mai ușor materia comparativ cu metodele exclusiv verbale
% mindomo.com
% .)
% Studiu pilot OECD – instrumente digitale și învățare (Germania, 2022): un proiect coordonat de OECD cu 730 de elevi (15 ani) a investigat utilizarea controlată a tehnologiei la clasă, incluzând activități interactive (de ex. simularea impactului turismului asupra unui ecosistem marin, programarea unui robot virtual etc.). Rezultatele au arătat că elevii care au folosit frecvent instrumente digitale în procesul de învățare au obținut scoruri cu până la 15% mai mari la testele de evaluare a noilor competențe (module pilot pentru PISA 2025) față de cei mai puțin familiarizați cu tehnologia
% oecdedutoday.com
% . Efectele benefice s-au evidențiat mai ales la capitole precum gândirea computațională și învățarea autoreglată
% oecdedutoday.com
% . În plus, sondajele atitudinale au relevat că peste 70% dintre elevi consideră instrumentele digitale ca fiind un ajutor în înțelegerea și reținerea conținuturilor dificile – tehnologia făcând învățarea mai ușoară, mai interesantă și facilitând memorarea pe termen lung
% oecdedutoday.com
% . Aceste constatări sugerează că, atunci când sunt folosite în mod adecvat și integrat în lecție, tehnologiile interactive pot spori eficiența educației, complementând metodele tradiționale și adresând diferitele nevoi de învățare ale elevilor
% oecdedutoday.com
% oecdedutoday.com
% .
% Prin urmare, dovezile de mai sus confirmă că vizualizările 3D, simulările și instrumentele interactive pot îmbunătăți semnificativ înțelegerea și retenția cunoștințelor comparativ cu predarea convențională. Elevii nu doar obțin rezultate mai bune la teste, dar își păstrează cunoștințele pentru mai mult timp și se declară mai satisfăcuți de procesul de învățare. Aceste beneficii se manifestă de la nivel gimnazial și liceal până la învățământul universitar, indicând un potențial larg de aplicare a tehnologiilor educaționale interactive în curriculumul STEM. Important de subliniat este că succesul depinde de utilizarea strategică a acestor instrumente – integrate cu un scop clar pedagogic, nu ca distracții, și însoțite de suport pentru profesori în proiectarea activităților. Folosite astfel, noile tehnologii devin un catalizator pentru o învățare mai aprofundată, mai activă și orientată spre secolul XXI.
% Statistici despre stilurile de învățare ale elevilor (accent pe stilul vizual)
% Un argument important pentru introducerea metodelor vizuale în educație este faptul că mulți elevi preferă în mod natural să învețe vizual. La nivel internațional, se vehiculează adesea că majoritatea populației are un stil de învățare preponderent vizual (imagini, diagrame), restul fiind în principal auditivi (sunete, explicații orale) sau kinestezici (învățare prin practică). Diverse surse indică aproximativ 65% persoane cu preferință vizuală, 30% auditivă și doar 5% tactil-kinestezică
% revistaprofesorului.ro
% . Acest raport de aprox. 2:1 în favoarea stilului vizual este citat pe scară largă în literatura educațională populară și sugerează că suporturile grafice sunt esențiale pentru o mare parte din elevi. În realitate, distribuția stilurilor de învățare poate varia în funcție de populație și context. De pildă, un studiu asupra elevilor din ciclul primar și secundar (Miller, 2001) a găsit proporții mai echilibrate: ~29% dintre elevi orientați vizual, 34% auditiv și 37% kinestezic
% iteach.ro
% . Chiar și la nivel regional pot apărea diferențe. Un exemplu local: într-o școală din România (221 de elevi chestionați), s-a constatat că 48% aveau stil de învățare vizual, 40% auditiv, iar 12% tactil-kinestezic
% revistaprofesorului.ro
% . Acest rezultat (vizual ~48%) este mai scăzut decât estimarea globală de 60-65%, lucru pe care autorii îl pun pe seama predominanței metodelor tradiționale verbale în școlile noastre – elevii fiind obișnuiți să învețe ascultând profesorul, nu lucrând cu imagini
% revistaprofesorului.ro
% . Totuși, vizualul rămâne stilul predominant și în acel eșantion, evidențiind necesitatea ca materialele didactice să includă componente vizuale pentru a atinge eficient aproape jumătate dintre elevi. 

% Distribuția procentuală a stilurilor de învățare – comparație între estimările globale (conform literaturii) și rezultatele unui eșantion de elevi din România. Se observă ponderea majoritar vizuală (culoare galbenă) la nivel global (~65%
% revistaprofesorului.ro
% ), comparativ cu ~48% vizuali în eșantionul local (culoare portocalie)
% revistaprofesorului.ro
% . (Sursa datelor: Revista Profesorului, 2020; iTeach.ro) În general, statisticile confirmă că stilul vizual de învățare este foarte răspândit în rândul copiilor și adolescenților. Elevii de astăzi sunt înconjurați de media vizuală (Internet, video, jocuri interactive) și astfel adesea gândesc și învață în imagini. Un sondaj al OECD a arătat că peste 70% dintre elevi simt că tehnologia vizuală îi ajută să învețe mai ușor conținut dificil
% oecdedutoday.com
% . Pentru profesori, aceste date sugerează importanța adaptării predării: utilizarea de diagrame, clipuri video, animații, modele 3D sau experimente virtuale poate activa stilul de învățare vizual al unui număr mare de elevi, crescând eficiența transmiterii informației. Desigur, specialiștii recomandă adoptarea unui echilibru – îmbinarea elementelor vizuale cu explicațiile verbale și activitățile practice – evitând supraspecializarea pe un singur stil de învățare
% revistaprofesorului.ro
% . Fiecare elev are un profil unic și adesea folosește o combinație de stiluri (vizual + auditiv, sau auditiv + practic etc.), astfel că o abordare didactică mixtă, care include prezentări vizuale alături de discuții și aplicații practice, va adresa mai bine diversitatea claselor de elevi. Concluzie: Integrând vizualizări 3D și tehnologii interactive în predarea STEM, profesorii pot valorifica predispoziția majorității elevilor către învățarea vizuală, pot facilita înțelegerea conceptelor complexe și pot îmbunătăți retenția cunoștințelor. Atât studiile academice, cât și datele din teren susțin eficiența acestor metode moderne, mai ales atunci când sunt folosite complementar metodelor tradiționale. Proiectul VisioScience3D se înscrie tocmai în această direcție, oferind resurse educaționale inovative menite să transforme materiile STEM în experiențe de învățare mai vizuale, interactive și accesibile pentru noua generație de elevi. Bibliografie selectivă: studii și rapoarte relevante în limbile română și engleză, de la organisme internaționale (OECD, UNESCO) și publicații academice, au fost utilizate pentru fundamentarea acestor concluzii
% stemeducationjournal.springeropen.com
% oecdedutoday.com
% revistaprofesorului.ro
% , alături de surse locale din domeniul educației
% revistaprofesorului.ro
% , asigurând o perspectivă cuprinzătoare asupra rolului vizualizărilor 3D în educația contemporană.


% Inline Listing example
% \lstset{language=make,caption=Application Makefile,label=lst:app-make}
% \begin{lstlisting}
% CSRCS = app.c
% SRC_DIR =..
% include $(SRC_DIR)/config/application.cfg
% \end{lstlisting}

% \begin{center}
% \begin{table}[htb]
%   \caption{Generated reports - associated Makefile targets and scripts}
%   \begin{tabular}{l*{6}{c}r}
%     Generated report & Makefile target & Script \\
%     \hline
%     Full Test Specification & full_spec & generate_all_spec.py  \\
%     Test Report & test_report & generate_report.py  \\
%     Requirements Coverage & requirements_coverage &
%     generate_requirements_coverage.py   \\
%     API Coverage & api_coverage & generate_api_coverage.py  \\
%   \end{tabular}
%   \label{table:reports}
% \end{table}
% \end{center}


\section{Soluția propusă}
\label{sec:proj}

\section{Rezultate obținute}
\label{sec:proj}

\section{Rezultate și concluzii pe scurt}
\label{sec:proj}

