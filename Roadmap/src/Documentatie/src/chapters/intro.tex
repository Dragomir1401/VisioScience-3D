\chapter{Introducere}
\label{chapter:intro}

% \textbf{This is just a demo file. It should not be used as a sample for a thesis.}\\
% \todo{Remove this line (this is a TODO)}

\section{Context}
\label{sec:proj}
% spatiu intre liinii
Educația este un domeniu de bază al socitetății, iar tehnologia joacă un rol din ce în ce mai important 
în acest sector. În special, într-o lume în care platformele sociale si mediul de interacțiune video subsection
de bază pentru tineri, este esențial să se dezvolte soluții educaționale care să fie atractive și eficiente în 
procesul clasic de învățare.

\subsection{Definirea problemei}
\label{sub-sec:proj-scope}
În acest context, problema pe care o abordăm este crearea unei platforme educaționale interactive care să integreze
tehnologii moderne și simulări 3D, pentru a face din procesul de învățare o experiență captivantă și eficientă pentru
elevi de gimnaziu și liceu. Această platformă va permite accesul vizual și facil la informații complexe de matematică,
fizică, chimie, astronomie și informatică.

Studenții vor putea explora concepte și interacționa cu simulări 3D, dar și să participe la teste și evaluări
pentru a-și verifica cunoștințele. De asemenea, profesorii vor avea la dispoziție un instrument pentru a crea teste și
a gestiona clasele de elevi, facilitând astfel procesul de predare și evaluare. Platforma va avea un sistem interactiv de 
navigare, recunoaștere a rezultatelor si răsplătirea progresului prin gamificare tot prin interacțiune 3D, ceea ce va
îmbunătăți experiența utilizatorilor și va stimula învățarea activă.


\subsection{Obiective}
\label{sub-sec:proj-objectives}

Obiectivele principale ale acestui proiect sunt:
\begin{itemize}
    \item Crearea unei platforme educaționale interactive care să integreze simulări 3D și tehnologii moderne.
    \item Dezvoltarea unui sistem de gestionare a testelor și evaluărilor pentru profesori și elevi.
    \item Implementarea unui sistem de gamificare pentru a stimula învățarea activă și implicarea utilizatorilor.
    \item Asigurarea accesibilității și ușurinței în utilizare pentru elevi și profesori.
    \item Crearea unui mediu de învățare captivant și eficient care să faciliteze înțelegerea conceptelor complexe.
    \item Integrarea unui sistem de raportare și monitorizare a progresului utilizatorilor.
    \item Crearea unei interfețe prietenoase și intuitive care să faciliteze experiența profesorilor în gestionarea
     claselor, testelor, elevilor și a rezultatelor.
\end{itemize}

% \fig[scale=0.5]{src/img/reporting-framework.pdf}{img:report-framework}{Reporting Framework}


\subsection{Susținere științifică}
\label{sub-sec:proj-scientific-support}

Multe discipline STEM implică concepte abstracte foarte dificil de vizualizat, ceea ce poate scădea interesul elevilor.
De exemplu, chimia este adesea percepută ca “prea abstractă” deoarece elevii nu pot vizualiza ușor concepte precum
structura moleculară sau reacțiile chimice. 

Integrarea vizualizărilor 3D și a tehnologiilor interactive în predarea disciplinelor STEM este susținută de un
număr semnificativ de cercetări recente. Acestea demonstrează că reprezentările vizuale și simulările contribuie
la înțelegerea conceptelor abstracte și sporesc motivația elevilor.

De exemplu, un studiu derulat în școlile din Cehia a arătat că utilizarea modelelor 3D și animațiilor în predarea
științelor a dus la o creștere semnificativă a implicării elevilor și a performanțelor la teste,
în special în chimie și biologie [13]. De asemenea, o meta-analiză recentă a concluzionat că lecțiile
care includ modele 3D interactive au îmbunătățit de peste 1,6 ori varianta standard de învățare teoretică [16].

Simulările 3D aplicate în laboratoare școlare au condus nu doar la o înțelegere mai bună a subiectelor,
ci și la o retenție îmbunătățită a cunoștințelor în timp [17]. Elevii au raportat un nivel mai ridicat
de încredere în propriile abilități și o atitudine mai pozitivă față de învățare.

Mai mult, numeroase cercetări evidențiază importanța predării adaptate stilurilor de învățare.
Datele arată că un procent semnificativ dintre elevi învață predominant vizual, ceea ce
justifică utilizarea elementelor grafice și a animațiilor în clasă [14], [15]. 
Un studiu local desfășurat în România confirmă această tendință, indicând o pondere
de aproximativ 48\% pentru stilul vizual, ceea ce subliniază necesitatea diversificării 
suportului educațional [14].

În plus, un raport OECD a demonstrat că utilizarea controlată a tehnologiei digitale
în procesul educațional poate conduce la o creștere cu până la 15\% a scorurilor
obținute de elevi la testele de competențe, comparativ cu metodele clasice [19].

Ca și concluzie, dovezile sugerează că integrarea vizualizărilor 3D și a simulărilor
interactive nu doar crește atractivitatea învățării, ci și eficiența ei.
Proiectul \textit{VisioScience3D} se aliniază acestor direcții moderne de predare,
oferind resurse educaționale inovative care răspund nevoilor noilor generații de elevi.



% Inline Listing example
% \lstset{language=make,caption=Application Makefile,label=lst:app-make}
% \begin{lstlisting}
% CSRCS = app.c
% SRC_DIR =..
% include $(SRC_DIR)/config/application.cfg
% \end{lstlisting}

% \begin{center}
% \begin{table}[htb]
%   \caption{Generated reports - associated Makefile targets and scripts}
%   \begin{tabular}{l*{6}{c}r}
%     Generated report & Makefile target & Script \\
%     \hline
%     Full Test Specification & full_spec & generate_all_spec.py  \\
%     Test Report & test_report & generate_report.py  \\
%     Requirements Coverage & requirements_coverage &
%     generate_requirements_coverage.py   \\
%     API Coverage & api_coverage & generate_api_coverage.py  \\
%   \end{tabular}
%   \label{table:reports}
% \end{table}
% \end{center}


\section{Soluția propusă}
\label{sec:proj}
Ideea platformei este de a crea un mediu de învățare interactiv care să integreze
simulări 3D și tehnologii moderne pentru a face procesul de învățare mult mai ușor.
Oferă o gamă de materii care pot fi studiate în această metodă inovativă, dar poate 
funcționa și ca verifcator ad-hoc al cunoștințelelor elevilor. Un elev poate intra rapid
și facil să verifice o formulă sau altă informație, iar profesorul poate să creeze teste
și să gestioneze clasele de elevi în mod rapid și eficient.

Numele VisioScience3D a fost ales pentru a reflecta scopul platformei și este compus din două
cuvinte: "Visio" care se referă la vizual, vedere, iar "Science" care se referă la știință.
Această combinație sugerează o platformă care îmbină ideea de vizual cu știința, oferind un nume
care reflectă esența platformei și scopul său. Titlul contine și termenul 3D, care subliniază
focusul vizualizărilor din platformă care sunt realizate tridimensional.

Fiecare rol din procesul educațional (elev, profesor) are posibilitatea de accesa funcționalitățile 
de învațăre și evaluare. De asemenea, platforma va avea un sistem de gamificare care va recompensa
elevii pentru progresul lor și va încuraja participarea activă. Opțiuni de vizualizare a tabele
de rezultate vor fi disponibile pentru profesori, iar elevii vor putea să-și urmărească scorul și 
progresul în timp real.

Testele pot fi create prin drag-and-drop în interfața secțiunii de create, unde exist control granular
de la nivel de structura a quizului până la nivel de întrebare, răspuns, selecție de imagini sau număr de 
răspunsuri corecte. Profesorii pot vizualiza clasele pe care le dețin, elevii care au participat la teste
și rezultatele obținute de aceștia. De asemenea, profesorii pot vizualiza și analiza rezultatele elevilor
pentru a înțelege mai bine progresul acestora și pentru a adapta metodele de predare în funcție de nevoile
fiecărui elev. Aceasta va permite o abordare personalizată a învățării, care poate îmbunătăți semnificativ
rezultatele elevilor. Profesorii au access si la sistemul de invitație a elevilor în platformă și în clasă
direct în contul elevului.

Elevii pot accesa platforma printr-o interfață prietenoasă și intuitivă, unde pot explora concepte
complexe prin simulări 3D și animații interactive. Pentru ei este destinat meniul 3D principal de selecție a 
materiei. unde pot vedea și interacționa cu toată gama de simulări disponibile. De asemenea, după cum am menționat
mai sus, elevii pot participa la teste și evaluări pentru a-și verifica cunoștințele. Aceste teste sunt concepute
pentru a fi interactive și captivante, oferind o experiență de învățare plăcută și eficientă, dar fiind și potrivite 
ad-hoc pentru o testare rapidă după o lecție predată.

\section{Rezultate obținute}
\label{sec:proj}

Platforma a ajuns intr-un punct în care poate fi utilizată de către profesori și elevi pentru a explora concepte
și reprezintă o soluție care poate salva timp și poate face învațarea mai rapidă și intuitivă pentru elevii
cu stil de învățare vizual, care după cum am menționat și după cum arată studiile sunt majoritari în școlile
din România (peste 48\% din elevi). La nivelul de profesor reprezintă curent o soluție rapidă de testare știința
monitorizare a elevilor la materii de știința, dar și de informatică.

La nivel tehnic, platforma folosește o arhitectură scalabilă a serviciilor din back-end, oferind o scalabilitate
foarte ridicată și o disponibilitate crescută datorită separării în microservicii a aplicației. Arhitectura poate
fi ușor extinsă pentru a adăuga noi funcționalități și module, iar platforma poate fi adaptată rapid la nevoile
utilizatorilor. De asemenea, partea de front-end este construită folosind tehnologii cu suport extins și
comunități mari, ceea ce asigură o suport îndelungat și o posibilă dezvoltare ușoară a platformei în viitor. 

\section{Structura lucrării}
\label{sec:proj}


Această lucrare este structurată în mai multe capitole, fiecare abordând un aspect diferit al proiectului atât din 
punct de vedere tehnic, cât și din punct de vedere al implementării și utilizării platformei.

Capitolul \ref{chapter:analiza} oferă o analiză detaliată a cerințelor și a nevoilor utilizatorilor, precum și a
profilului utilizatorilor tipici ai platformei. Acesta include o descriere a motivației, a cerințelor funcționale și
a celor nefuncționale, precum și a limitărilor și constrângerilor proiectului.

Al treilea capitol (\ref{chapter:studiuPiata}) analizează piața și competiția din punct de vedere al platformelor educaționale
existente, evidențiind punctele forte și slabe ale acestora și modul în care platforma propusă se diferențiază
de cea dezvoltată în cadrul acestui proiect. 

Capitolul \ref{chapter:tehnologii} detaliază tehnologiile utilizate în dezvoltarea platformei, inclusiv limbajele de
programare, framework-urile și instrumentele utilizate pentru a crea aplicația. Aici se oferă o privire de ansamblu 
asupra dezvoltării fiecarei părți importante a aplicației: back-end, front-end (inclusiv UI/UX), scene 3D și bazele de date.


Capitolul \ref{chapter:implementare} oferă detalii despre implementarea platformei, inclusiv bucăți de cod și exemple De
dezvoltare a codului, precum și detalii despre configurarea și conectivitatea obținută între diferitele componente 
ale aplicației.

Cel de-al șaselea capitol, \ref{chapter:scenariiUtilizare} are ca focus definirea și prezentarea scenariilor de utilizare
ale platformei, inclusiv modul în care utilizatorii pot interacționa cu aplicația și cum pot crea conturi, teste, clase
și cum pot vizualiza scene și experimente 3D. De asemenea, se discută despre modul în care utilizatorii pot accesa și utiliza
funcționalitățile platformei, precum și despre modul în care pot beneficia de gamificare și recompense pentru progresul
lor.

Capitolul \ref{chapter:evaluare} se concentrează pe tipurile de evaluare a platformei, cu accent pe testare și pe modul în
care se poate monitoriza sănătatea platformei în cazul lansării către publicul larg. De asemenea se discută și despre
evaluare performanței platformei pe diferite niveluri.


Ultimul capitol, \ref{chapter:concluzii}, oferă concluzii și perspective asupra viitorului platformei, inclusiv
posibile îmbunătățiri și extinderi ale funcționalităților existente. De asemenea, se discută despre impactul pe care
platforma ar putea să-l aibă asupra educației.