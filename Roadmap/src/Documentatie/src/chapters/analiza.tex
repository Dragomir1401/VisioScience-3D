\chapter{Analiza cerințelor / Motivația proiectului}
\label{chapter:analiza}

\section{Plaja de utilizatori}
\label{sec:proj}

Pulicul țintă al aplicației este destul de general, putând fi accesată de orice
utilizator care se înregistrează și dorește să învețe sau sa iși amintească noțiuni de matematică, fizică, chimie,
astronomie sau informatică.

\subsection{Categorii}
\label{sub-sec:proj-scope}

Categoriile principale de utilizatori suportate de platformă sunt:
\begin{itemize}
    \item Utilizatori elevi
    \item Utilizatori profesori
\end{itemize}

Aceste categorii indică și publicul țintă al aplicației, care este format din elevi de nivel gimnaziu sau liceu
și profesori care predau disciplinele menționate la acest nivel de învățământ.

Între aceste două categorii de utilizatori diferă drepturile de acces dar și tipul de interacțiune cu aplicația dar și
functionalității disponibile.

\subsection{Profilul utilizatorului}
\label{sub-sec:proj-user-profile}

După cum am menționat anterior, aplicația este destinată elevilor și profesorilor de matematică, fizică, chimie, astronomie și informatică.
Putem face și o analiză din diverse puncte de vedere demografice și comportamentaleȘ
\begin{itemize}
    \item \textbf{Vârstă}: 10-18 ani pentru elevi, 25-60 ani pentru profesori

    Motivat de faptul că aplicația este destinată elevilor de gimnaziu și liceu.
    \item \textbf{Tip de învațăre}: vizual

    Justificat de faptul că aplicația folosește animații 3D pentru a explica concepte științifice.
    \item \textbf{Studii}: elevi de gimnaziu și liceu, profesori cu studii superioare în domeniul educației sau al științelor exacte
    
    Motivat de scopul și ținta aplicației.
    \item \textbf{Locație}: România, limba destinată fiind româna

    Justificat de faptul că aplicația este destinată elevilor și profesorilor din România.
    \item \textbf{Interese}: educație, tehnologie, știință, învățare interactivă
    
    Justificat de faptul că are suport pentru materii de știință și tehnologie.
    \item \textbf{Motivație elevi}: dorința de a învăța și de a-și îmbunătăți cunoștințele în domeniile menționate, dorința de a obține note mai mari la școală
    
    Motivat de faptul că aplicația este destinată elevilor care doresc să învețe și să-și îmbunătățească cunoștințele.
    \item \textbf{Motivație profesori}: dorința de a-și îmbunătăți metodele de predare, dorința de a oferi elevilor o experiență de 
    învățare mai interactivă și mai captivantă

    Justificat de faptul că aplicația este destinată profesorilor care doresc să-și îmbunătățească metodele de predare.
    \item \textbf{Tehnologie}: utilizatori cu un nivel cel puțin începător-mediu de competență tehnologică, familiarizați cu utilizarea aplicațiilor web
    
    Datorită faptului că aplicația este o aplicație web care necesită cunoștințe de bază în accesare și navigarea pe internet.
    \item \textbf{Dispozitive}: utilizatori care folosesc computere, laptopuri, tablete sau telefoane mobile pentru accesarea aplicației
    
    Motivat de faptul că aplicația este o aplicație web care poate fi accesată de pe orice dispozitiv cu acces la internet.
\end{itemize}

\section{Motivația proiectului}
\label{sec:proj-motivation}

Motivația proiectului este de a oferi o platformă educațională interactivă care să ajute elevii, decizia ideii fiind luată după ce s-a studiat
piața de soluții existente care oferă acest tip de produs destinat României, în limba română și care oferă o experiență modernă, cât și suport pentru 
testare incorporat. 

\section{Cerințe funcționale}
\label{sec:proj-requirements}

interfața a fost realizată după două principii de bază:
\begin{itemize}
    \item \textbf{Simplitate}: am ales un design simplu, minimalist, care să nu distragă atenția utilizatorului de la conținutul educațional
    \item \textbf{Interactivitate}: am ales să folosim animații 3D în cât mai multe zone ale aplicației, pentru a face experiența de învățare mai captivantă și mai plăcută
    \item \textbf{Accesibilitate}: am ales să folosim o paletă de culori care să fie ușor de citit și să nu obosească ochii utilizatorului
\end{itemize}

Cerințele funcționale sunt împărțite în două mari categorii, în funcție de tipul de utilizator:
\begin{itemize}
    \item Cerințe funcționale pentru utilizatorii elevi
    \item Cerințe funcționale pentru utilizatorii profesori
    \item Cerințe funcționale pentru sistem
\end{itemize}

\subsection{Cerințe funcționale pentru utilizatorii elevi}
\label{sub-sec:proj-requirements-students}

\begin{itemize}
    \item Utilizatorii elevi trebuie să se poată înregistra în aplicație
    \item Utilizatorii elevi trebuie să se poată autentifica în aplicație
    \item Utilizatorii elevi trebuie să aibă acces la un meniu principal 3D de selectare a materiei
    \item Utilizatorii elevi trebuie să aibă acces la secțiuni educaționale
    \item Utilizatorii elevi trebuie să aibă acces la scenele 3D și să poată interacționa cu ele
    \item Utilizatorii elevi trebuie să poată intra în clase create de profesori prin invitație
    \item Utilizatorii elevi trebuie sa poată răspundă la invitații
    \item Utilizatorii elevi trebuie să aibă acces să vizualizeze testele deschise
    \item Utilizatorii elevi trebuie să aibă să rezolve testele deschise
    \item Utilizatorii elevi trebuie să aibă acces la rezultatele obținute dacă profesorul a ales să le facă publice
    \item Utilizatorii elevi trebuie să aibă acces profilul de utilizator de gestiune a contului
\end{itemize}

\subsection{Cerințe funcționale pentru utilizatorii profesori}
\label{sub-sec:proj-requirements-teachers}

\begin{itemize}
    \item Utilizatorii profesori trebuie să se poată înregistra în aplicație
    \item Utilizatorii profesori trebuie să se poată autentifica în aplicație
    \item Utilizatorii profesori trebuie să aibă acces la un meniu principal 3D de selectare a materiei
    \item Utilizatorii profesori trebuie să aibă acces la secțiuni educaționale
    \item Utilizatorii profesori trebuie să aibă acces la scenele 3D și să poată interacționa cu ele
    \item Utilizatorii profesori trebuie să aibă acces la un meniu de gestionare a elevilor și claselor
    \item Utilizatorii profesori trebuie să aibă acces la un meniu de gestionare a testelor
    \item Utilizatorii profesori trebuie să aibă acces la un meniu de vizualizare a rezultatelor obținute de elevi
    \item Utilizatorii profesori trebuie să aibă acces profilul de utilizator de gestiune a contului
    \item Utilizatorii profesori trebuie să aibă acces la un meniu de gestionare a invitațiilor
    \item Utilizatorii profesori trebuie să aibă acces să genereze invitații pentru elevi
    \item Utilizatorii profesori trebuie să aibă acces să trimită invitații pentru elevi
    \item Utilizatorii profesori trebuie să poată vedea și gestiona rezultatele elevilor
    \item Utilizatorii profesori trebuie să poată crea teste și închide/deschide accesul la ele
\end{itemize}

\subsection{Cerințe funcționale pentru sistem}
\label{sub-sec:proj-requirements-system}

\begin{itemize}
    \item Sistemul trebuie să trimită invitațiile aproape instantaneu către utilizatori
    \item Sistemul trebuie să trimită notificări utilizatorilor atunci când primesc invitații
    \item Sistemul trebuie să încarce rapid scenele 3D
    \item Sistemul trebuie să aibă un timp de răspuns rapid la interacțiunile utilizatorilor
    \item Sistemul trebuie să răspundă instant la interacțiunile utilizatorilor cu scenele 3D
    \item Sistemul trebuie să ofere rezultatele la teste instant
    \item Sistemul trebuie să gestioneze eficient sistemul de utilizatori și autentificare
    \item Sistemul trebuie să gestioneze sistemul de închidere/deschidere a testelor
\end{itemize}


\section{Cerințe nefuncționale}
\label{sec:proj-non-functional-requirements} 

\section{Limitări}
\label{sec:proj-limitations}
