\chapter{Tehnologii utilizate pentru front-end}
\label{chapter:tehnologii}

\label{sec:proj}
\section{Soluția de front-end}
\subsection{Framework principal}

Frameworkul principal pe care l-am folosit a fost React JS, care este o bibliotecă foarte populară
și răspândită de JavaScript. În evaluarea framework-urilor front-end pentru VisioScience3D,
 ne-am concentrat pe criterii precum maturitatea ecosistemului, performanță, curba de învățare,
  flexibilitate în personalizare şi capacitatea de integrare cu biblioteci 3D. Dintre principalele
   opțiuni (Vue.js, Angular, Svelte), React s-a impus datorită următoarelor avantaje:

Ecosistem matur și susținere largă: React are o comunitate activă de milioane de dezvoltatori,
 cu un număr impresionant de librării, tutoriale și un suport consistent din partea Meta. 
 Această resursă ne-a permis să adoptăm rapid bune practici și să găsim soluții la provocări 
 specifice dezvoltării 3D.

Model declarativ și component-driven: React oferă un mod declarativ de descriere a interfeței,
 ideal pentru scene 3D complexe, unde starea se propagă predictibil prin componente. Aceasta 
 ușurează managementul ciclului de viață al obiectelor Three.js, reducând riscul de memory 
 leaks sau inconsistent updates.

React Router DOM: Pentru navigarea între diferitele secţiuni ale platformei (fizică, chimie,
 astronomie etc.), React Router DOM a fost ales pentru simplitatea în definirea “routes” şi 
 suportul pentru lazy loading de module. Alternativele (Vue Router, Angular Router) sunt la 
 fel de capabile, însă integrarea lor cu React Three Fiber ar fi impus un strat suplimentar
  de interoperabilitate.

De ce Tailwind CSS vs. alte soluții de stilizare

Pentru sistemul de design al VisioScience3D, am comparat framework-uri clasice de CSS
 (Bootstrap, Material UI) cu Tailwind CSS:

Utilitare-first și JIT: Tailwind oferă clase utilitare care permit definirea rapidă a 
layout-urilor şi a stilurilor unice, fără a scrie sutele de linii de CSS personalizat. Motorul 
JIT (Just-In-Time) generează doar clasele folosite, menținând pachetul final foarte mic.

Consistență și flexibilitate: În loc să ne încărcăm aplicația cu componente predefinite,
 am putut construi ună interfață coerentă, respectând paleta de culori şi spațierile dorite,
  fără a rescrie componente UI complexe.

Integrare strânsă cu React: Utilizarea className din JSX se potrivește natural cu Tailwind,
 iar plugin-urile precum @tailwindcss/forms facilitează stilizarea formularelor şi input-urilor.

Beneficiile integrării cu Three.js prin React Three Fiber și Drei

Pentru vizualizările 3D ale conceptelor fizice și chimice, am ales React Three Fiber (R3F) şi Drei:

Abstracție declarativă: R3F “împachetează” API-ul imperativ Three.js într-un DSL React, 
permițându-ne să definim scene, obiecte și materiale în JSX, sub formă de componente. 
Astfel, putem folosi hook-uri (useFrame, useLoader) pentru a sincroniza animațiile și 
resursele fără boilerplate.

Ecosistem Drei: Biblioteca Drei aduce peste 100 de componente gata făcute (OrbitControls,
 Text, PositionalAudio etc.), accelerate de comunitate, care ne-au economisit ore de 
 implementare. De exemplu, Text pentru etichete 3D sau Html pentru overlay-uri de UI.

Optimizări de performanță: Datorită reconciler-ului React, R3F actualizează doar părțile
 din scenă care se schimbă. Împreună cu dpr configurabil și performance API, am putut regla
  dinamic calitatea redării pe dispozitive diverse, fără compromisuri semnificative.

În concluzie, combinația React + React Router DOM + Tailwind CSS + React Three Fiber + 
Drei ne-a oferit un stack unificat, performant și ușor de întreținut, perfect adaptat 
pentru dezvoltarea rapidă de simulări 3D interactive în browser și pentru extinderea
platformei VisioScience3D.

\fig[width=1.1\textwidth]{imgs/fediagram.png}{fig:mozaweb}{Diagrama de design primncipală a frontend-ului}

\subsection{Framework 3D}

\section{Soluția UI/UX}
\label{sub-sec:proj-ui-ux}
\subsection{Brandingul proiectului}
\subsubsection{Paleta de culori}

\subsection{Experiența utilizatorului}
\subsubsection{Crearea componentelor UI}
\subsubsection{Designul interfeței utilizatorului}

\section{Soluția de creare a scenelor 3D}
\label{sec:proj-3d}
\subsection{Utilizarea tehnologiilor WebGL și Three.js}
\subsubsection{Crearea și gestionarea scenelor 3D}
\subsubsection{Interacțiunea cu obiectele 3D}
\subsubsection{Optimizarea performanței graficii 3D}
\subsection{Integrarea cu frontendul}







