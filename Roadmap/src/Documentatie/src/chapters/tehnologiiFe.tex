\chapter{Tehnologii utilizate pentru front-end}
\label{chapter:tehnologii}

\label{sec:proj}
\section{Soluția de front-end}
\subsection{Framework principal}

Frameworkul principal pe care l-am folosit este React JS, care este o tehnologie foarte populară
și răspândită bazată pe JavaScript. În alegerii de tehnologie pentru front-endul platformei VisioScience3D,
am concentrat atenția pe criterii precum maturitatea ecosistemului, performanță, curba de învățare,
flexibilitate în personalizare şi capacitatea de integrare cu biblioteci 3D. Dintre principalele
opțiuni (Vue.js, Angular, Svelte), React s-a impus datorită următoarelor avantaje:

\begin{itemize}
\item
  \textbf{Ecosistem matur și susținere largă}:
    React are o comunitate activă de milioane de dezvoltatori,
    cu un număr impresionant de librării, tutoriale și un suport consistent din partea Meta. 
    Această resursă ne-a permis să adoptăm rapid bune practici și să găsim soluții la provocări 
    specifice dezvoltării 3D.
\item
  \textbf{Model declarativ și component-driven}:
    React oferă un mod de declarare detaliată în descrierea interfeței, ideal pentru scene 3D
    complexe, unde starea se propagă predictibil prin componente. Aceasta ușurează managementul 
    ciclului de viață al obiectelor din biblioteca 3D, reducând riscul de actualizări inconsistente.

\item
  \textbf{React Router DOM}:
    Pentru că oferă o soluție de rutare foarte ușoara în navigarea între diferitele
    secţiuni ale platformei (fizică, chimie, astronomie etc.) prin React Router DOM 
    ce oferă simplitate în definirea rutelor şi suportul pentru incărcare întârziată de module.
    Alternativele (Vue Router, Angular Router) sunt la fel de capabile, însă integrarea lor cu
    React Three Fiber, care vom vedea este baza în simulările 3D, ar fi impus un strat suplimentar
    de interoperabilitate.
\end{itemize}


Pe zona de stilizare a interfeței, am optat pentru Tailwind CSS, un framework CSS utilitar 
care ne-a permis să creăm un design personalizat. Am comparat framework-uri clasice de CSS
(Bootstrap, Material UI) cu Tailwind CSS și am observat următoarele avantaje:
\begin{itemize}
\item
  \textbf{Utilitare-întâi și JIT}:
    Tailwind oferă clase utilitare care permit definirea rapidă a
    layout-urilor şi a stilurilor unice, fără a scrie sutele de linii de CSS personalizat. Motorul
    JIT (Just-In-Time) generează doar clasele folosite, menținând pachetul final mai mic.
\item
  \textbf{Consistență și flexibilitate}:
    În loc să ne încărcăm aplicația cu componente predefinite,
    am putut construi ună interfață coerentă, respectând paleta de culori şi spațierile dorite,
    fără a rescrie componente UI complexe.
\item
  \textbf{Integrare strânsă cu React}:
    Utilizarea className din JSX se potrivește natural cu Tailwind,
    iar plugin-urile precum @tailwindcss/forms facilitează stilizarea
    formularelor şi input-urilor.
\item
  \textbf{Extensibilitate}:
    Tailwind permite personalizarea ușoară a temelor, adăugarea de plugin-uri și crearea de 
    componente reutilizabile, fără a sacrifica viteza de dezvoltare.
\end{itemize}

Bibliotecile de creare și gestionare a graficii 3D sunt esențiale pentru platforma noastră,
iar pentru a crea simulări interactive, am ales Three.js, o bibliotecă JavaScript populară.
Three.js oferă un API puternic pentru crearea de scene 3D, animații și interacțiuni și Împreună
cu React Three Fiber (R3F), o bibliotecă care integrează Three.js cu React, am putut să ne concentrăm
pe dezvoltarea rapidă a aplicațiilor 3D fără a ne preocupa de detaliile de implementare ale Three.js.
Utilizarea peste a Drei, ce conține o suită de componente și utilitare pentru R3F, a permis accelerarea
procesului de dezvoltare, oferind soluții gata făcute pentru anumite probleme comune.

Câteva dintre beneficiile integrării cu Three.js prin React Three Fiber și Drei care au făcut 
alegerea justificată sunt:

\begin{itemize}
\item
  \textbf{Abstracție declarativă}:
    R3F “împachetează” API-ul imperativ Three.js într-un DSL React, 
    permițându-ne să definim scene, obiecte și materiale în JSX, sub formă de componente. 
    Astfel, putem folosi hook-uri (useFrame, useLoader) pentru a sincroniza animațiile și 
    resursele fără prea mult cod duplicat.
\item
  \textbf{Ecosistem Drei}:
    Biblioteca Drei aduce peste 100 de componente existente (OrbitControls,
    Text, etc.), accelerate de comunitate, care au economisit timp de
    implementare.
\item
  \textbf{Optimizări de performanță}:
    Datorită reconciler-ului React, R3F actualizează doar părțile
    din scenă care se schimbă.
\end{itemize}

Așadar, combinația React + React Router DOM + Tailwind CSS + React Three Fiber + Drei ne-a oferit un
stack unificat, performant și ușor de întreținut, bine adaptat
pentru dezvoltarea rapidă de simulări 3D interactive în browser și pentru extinderea ușoară a
platformei VisioScience3D. Se poate observa în figura \ref{fig:fediagram} cum arată
arhitectura simplificată a frontend-ului platformei.

\fig[width=1.1\textwidth]{imgs/fediagram.png}{fig:fediagram}{Diagrama de design principală a frontend-ului}

\subsection{Framework 3D}

După cum am discutat anterior, frameworkul 3D ales este React Three Fiber, care
este o bibliotecă care integrează Three.js cu React, permițându-ne să creăm scene 3D
într-un mod declarativ și reactiv. Aceasta se bazează pe principii de WebGL combinate
cu ce poate oferi React, cum ar fi componentizarea și gestionarea stării. R3F ne permite să
creăm scene 3D complexe folosind JSX și să integrăm hook-uri ca:

\begin{itemize}
\item
  \textbf{useFrame}:
    Acesta permite să actualizăm scena la fiecare cadru, facilitând animațiile și interacțiunile
    în timp real.
\item
  \textbf{useLoader}:
    Acesta ajută să încărcăm resurse externe (texturi, modele 3D) într-un mod eficient,
    folosind promisiuni pentru a gestiona încărcarea asincronă.
\item
  \textbf{useThree}:
    Acesta oferă acces la obiectul Three.js curent, permițându-ne să interacționăm direct
    cu scena, camera și renderer-ul.
\item
  \textbf{useRef}:
    Acesta permite să creăm referințe la obiectele din scenă, facilitând manipularea lor
    direct în timpul animațiilor sau interacțiunilor.
\item
  \textbf{useState}:
    Acesta permite să gestionăm starea componentelor, facilitând actualizarea și redarea
    dinamică a obiectelor din scenă.
\item
  \textbf{useEffect}:
    Acesta permite să gestionăm efectele secundare, cum ar fi adăugarea de evenimente sau
    actualizarea stării în funcție de modificările din scenă.
\end{itemize}

Pentru integrarea obiectelor cu extensia .glb am folosit un convertor gltf  și anume [20] care
ne permite să convertim modele .glb în componente React Three Fiber oferind grupuri de meshe-uri și
materiale. Acest lucru a permis să importăm modelele 3D direct în aplicația React și să creăm
diverse manipulări și animații folosind API-ul R3F.


\fig[width=1\textwidth]{imgs/convertor.png}{fig:fediagram}{Exemplu de convertire a unui model 3D .glb în R3F}


\subsection{Modelare 3D}
Am folosit Blender pentru a crea modelele 3D utilizând modele și primitive existente pe internet. De exemplu
modelul paginii principale a fost făcut din primitive de insule și castele și adaptat la un număr de 
insule potrivit materiilor din meniu, cât și loc pentru suport viitor pentru alte materii. 

Putem observa în figura următoare dezvoltarea modelului în Blender.

\fig[width=1\textwidth]{imgs/Blender.png}{fig:fediagram}{Modelul paginii principale în Blender}

La acest model s-au adăugat animații de rotație și schimbare a state-ului la mișcarea mouse-ului,
animație continuă de plutire și animații realiste de zbor pentru dronele din jurul insulelor.


\section{Soluția UI/UX}
\label{sub-sec:proj-ui-ux}
\subsection{Brandingul proiectului}
\subsubsection{Paleta de culori}
Brandingul VisioScience3D se bazează pe o paletă caldă, prietenoasă şi destul de pastelată,
menită să creeze o atmosferă de joc și luminoasă pentru elevi. Fundalurile folosesc un gradient
subtil de la \#fdf4ff (alb-roz pal) prin \#f3e8ff (lavandă delicată) spre \#fff7ed (piersică pal),
în timp ce culorile de accent dau personalitate interfeţei: nuanţa profundă de mulberry (\#690375)
marchează elementele active, bordurile şi textele importante, iar tonul rosy-brown (\#AE847E) 
individualizează secţiunile de chimie şi astronomie. Pentru evidenţierea vizualizărilor 3D
am introdus un violet închis (\#4f46e5). Culorile reflectă o atmosferă dinamică și 
energică, încurajând explorarea și învățarea.

\fig[width=1\textwidth]{imgs/paleta.png}{fig:fediagram}{Paleta de culori a platformei}

\subsubsection{Logo-ul}
Logo-ul VisioScience3D este relativ simplu și în temă cu paleta de culori, fiind reprezentat
de cuvantul "VisioScience3D" scris cu un font prezentativ și colorat cu un gradient de la
mov la mulberry (\#690375).

\fig[width=0.5\textwidth]{imgs/a.png}{fig:fediagram}{Logo-ul platformei}

\subsection{Experiența utilizatorului}
\subsubsection{Navigarea în platformă}
\subsubsection{Crearea componentelor UI}
\subsubsection{Designul interfeței utilizatorului}

\section{Soluția de creare a scenelor 3D}
\label{sec:proj-3d}
\subsection{Utilizarea tehnologiilor WebGL și Three.js}
\subsubsection{Crearea și gestionarea scenelor 3D}
\subsubsection{Interacțiunea cu obiectele 3D}
\subsubsection{Optimizarea performanței graficii 3D}
\subsection{Integrarea cu frontendul}







