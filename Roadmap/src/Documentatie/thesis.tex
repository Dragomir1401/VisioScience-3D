\documentclass{thesis.cs.pub.ro}
\makeindex
\begin{document}

% Thesis info
\title{VisioScience3D}
\author{Dragomir Andrei-Mihai}
\date{2025}

% \titleen{Magnificent Title}
% \titlero{Un titlu de senzație}
% \adviser{Prof. Magnificus Academicus}
% % \titlefooteren{Bucharest, Year of Grace}
% \titlefooterro{București, Anul de Grație}

% Macros
% example: \newcommand{\EU}{European Union}
% use '\EU\ ' if you want to enforce a space after the macro

\newcommand{\project}{MySuperProject}



\begin{frontmatter} % roman numbering

%\maketitle
\begin{titlepage}
	\begin{center}
		{\Large Universitatea POLITEHNICA din București}\\[2mm]
		{\Large Facultatea de Automatică și Calculatoare, Departamentul de Calculatoare}\\[3mm]

		\begin{tabular}{ccc}
			\includegraphics[width=0.13\textwidth]{src/img/branding/upb} & \hspace{3cm} & \includegraphics[width=0.30\textwidth]{src/img/branding/cs}
		\end{tabular}

		\vspace*{35mm}
		{\Huge LUCRARE DE DIPLOMĂ}\\[15mm]
		{\Huge VisioScience3D}\\[35mm]

		\begin{tabular}{p{0.45\textwidth} c p{0.45\textwidth}}
			\raggedright \Large \textbf{Conducător Științific:} \\ \Large Moraru Anca Andreea
			& &
			\raggedleft \Large \textbf{Autor:} \\ \Large Dragomir Andrei-Mihai
		\end{tabular}

		\vspace*{40mm}
		\Large București, 2025
	\end{center}
\end{titlepage}

\begin{titlepage}
	\begin{center}
		{\Large University POLITEHNICA of Bucharest}
		\par\vspace*{2mm}
		{\Large Faculty of Automatic Control and Computers,
		
		Computer Science and Engineering Department}
		\par\vspace*{3mm}
		\begin{table*}[h]
        	\begin{center}
				\begin{tabular}{cccc}
                    \includegraphics[width=0.13\textwidth]{src/img/branding/upb}
					& & &
					\includegraphics[width=0.30\textwidth]{src/img/branding/cs}
            	\end{tabular}
			\end{center}
		\end{table*}
		
		\par\vspace*{35mm}
		{\Huge BACHELOR THESIS}
		\par\vspace*{15mm}
		{\Huge VisioScience3D}
		\par\vspace*{35mm}
		\begin{table*}[h]
        	\begin{center}
				\begin{tabular}{lcccccl}
					\Large \textbf{\Large Scientific Adviser:}
					\vspace*{1mm} &&&&&& \Large \textbf{\Large Author:}\vspace*{1mm} \\
					\Large Moraru Anca Andreea &&&&&& \Large Dragomir Andrei-Mihai \\
				\end{tabular}
			\end{center}
		\end{table*}

		\par\vspace*{40mm}
		\Large Bucharest, 2025
	\end{center}
\end{titlepage}


\begin{acknowledgements}
% This file contains acknowledgements
\vspace*{7cm}
\begin{center}
Realizarea acestui proiect nu ar fi fost posibilă fără ajutorul și sprijinul\\
D-nei Prof.dr.ing Anca Andreea Moraru. Mulțumesc pentru îndrumare,\\
promptitudine și răbdare.\\
Mulțumesc tuturor celorlalți profesori, laboratori și colegi care mi-au facilitat\\
procesul de învățare și mi-au oferit suportul necesar de-a lungul acestor 4 ani, punându-mi
bazele solide in dimeniul ingineriei de Calculatoare.\\
\end{center}


\end{acknowledgements}

\begin{abstract}
% This file contains the abstract of the thesis

Proiectul VisioScience3D vine ca un răspuns la nevoia de a crea un mediu de învățare\\
interactiv și captivant pentru elevii din învățământul preuniversitar.\\
Acesta îmbină tehnologia avansată pentru a crea un sistem de învățare bazat pe vizualizări 3D\\
și simulări interactive, care să faciliteze înțelegerea conceptelor complexe din domeniul științelor exacte.\\




In momentul curent învățarea geometriei, fizicii, chimiei și altor discipline științifice\\
se face prin metode tradiționale, care nu reușesc să capteze atenția elevilor.\\
Prin acest proiect ne dorim sa venim în întâmpinarea acestei nevoi,\\
să oferim un mediu de învățare interactiv și captivant care să faciliteze\\
activitatea didactică și să îmbunătățească rezultatele elevilor.\\




Proiectul VisioScience3D este o aplicație web care permite utilizatorilor să exploreze\\
domeniul științelor exacte prin intermediul simulărilor interactive și vizualizărilor 3D.\\
Acesta oferă o soluție inovatoare si complxă pentru elevi și profesori,\\
avand posibilitate sa predea si evalueze elevii prin intermediul platformei.\\
\end{abstract}

\input{template/contents.pages}

\printabbrev

\end{frontmatter} % end roman numbering

% 1.5 linespacing for the contents
% \doublespacing 
\setstretch{1.6}  

% All chapters
% do not include .tex extension
\chapter{Introducere}
\label{chapter:intro}

% \textbf{This is just a demo file. It should not be used as a sample for a thesis.}\\
% \todo{Remove this line (this is a TODO)}

\section{Context}
\label{sec:proj}
% spatiu intre liinii
Educația este un domeniu de bază al socitetății, iar tehnologia joacă un rol din ce în ce mai important 
în acest sector. În special, într-o lume în care platformele sociale si mediul de interacțiune video subsection
de bază pentru tineri, este esențial să se dezvolte soluții educaționale care să fie atractive și eficiente în 
procesul clasic de învățare.

\subsection{Definirea problemei}
\label{sub-sec:proj-scope}
În acest context, problema pe care o abordăm este crearea unei platforme educaționale interactive care să integreze
tehnologii moderne și simulări 3D, pentru a face din procesul de învățare o experiență captivantă și eficientă pentru
elevi de gimnaziu și liceu. Această platformă va permite accesul vizual și facil la informații complexe de matematică,
fizică, chimie, astronomie și informatică.

Studenții vor putea explora concepte și interacționa cu simulări 3D, dar și să participe la teste și evaluări
pentru a-și verifica cunoștințele. De asemenea, profesorii vor avea la dispoziție un instrument pentru a crea teste și
a gestiona clasele de elevi, facilitând astfel procesul de predare și evaluare. Platforma va avea un sistem interactiv de 
navigare, recunoaștere a rezultatelor si răsplătirea progresului prin gamificare tot prin interacțiune 3D, ceea ce va
îmbunătăți experiența utilizatorilor și va stimula învățarea activă.


\subsection{Obiective}
\label{sub-sec:proj-objectives}

Obiectivele principale ale acestui proiect sunt:
\begin{itemize}
    \item Crearea unei platforme educaționale interactive care să integreze simulări 3D și tehnologii moderne.
    \item Dezvoltarea unui sistem de gestionare a testelor și evaluărilor pentru profesori și elevi.
    \item Implementarea unui sistem de gamificare pentru a stimula învățarea activă și implicarea utilizatorilor.
    \item Asigurarea accesibilității și ușurinței în utilizare pentru elevi și profesori.
    \item Crearea unui mediu de învățare captivant și eficient care să faciliteze înțelegerea conceptelor complexe.
    \item Integrarea unui sistem de raportare și monitorizare a progresului utilizatorilor.
    \item Crearea unei interfețe prietenoase și intuitive care să faciliteze experiența profesorilor în gestionarea
     claselor, testelor, elevilor și a rezultatelor.
\end{itemize}

% \fig[scale=0.5]{src/img/reporting-framework.pdf}{img:report-framework}{Reporting Framework}


\subsection{Susținere științifică}
\label{sub-sec:proj-scientific-support}

Multe discipline STEM implică concepte abstracte foarte dificil de vizualizat, ceea ce poate scădea interesul elevilor.
De exemplu, chimia este adesea percepută ca “prea abstractă” deoarece elevii nu pot vizualiza ușor concepte precum
structura moleculară sau reacțiile chimice. 

Integrarea vizualizărilor 3D și a tehnologiilor interactive în predarea disciplinelor STEM este susținută de un
număr semnificativ de cercetări recente. Acestea demonstrează că reprezentările vizuale și simulările contribuie
la înțelegerea conceptelor abstracte și sporesc motivația elevilor.

De exemplu, un studiu derulat în școlile din Cehia a arătat că utilizarea modelelor 3D și animațiilor în predarea
științelor a dus la o creștere semnificativă a implicării elevilor și a performanțelor la teste,
în special în chimie și biologie [13]. De asemenea, o meta-analiză recentă a concluzionat că lecțiile
care includ modele 3D interactive au îmbunătățit de peste 1,6 ori varianta standard de învățare teoretică [16].

Simulările 3D aplicate în laboratoare școlare au condus nu doar la o înțelegere mai bună a subiectelor,
ci și la o retenție îmbunătățită a cunoștințelor în timp [17]. Elevii au raportat un nivel mai ridicat
de încredere în propriile abilități și o atitudine mai pozitivă față de învățare.

Mai mult, numeroase cercetări evidențiază importanța predării adaptate stilurilor de învățare.
Datele arată că un procent semnificativ dintre elevi învață predominant vizual, ceea ce
justifică utilizarea elementelor grafice și a animațiilor în clasă [14], [15]. 
Un studiu local desfășurat în România confirmă această tendință, indicând o pondere
de aproximativ 48\% pentru stilul vizual, ceea ce subliniază necesitatea diversificării 
suportului educațional [14].

În plus, un raport OECD a demonstrat că utilizarea controlată a tehnologiei digitale
în procesul educațional poate conduce la o creștere cu până la 15\% a scorurilor
obținute de elevi la testele de competențe, comparativ cu metodele clasice [19].

Ca și concluzie, dovezile sugerează că integrarea vizualizărilor 3D și a simulărilor
interactive nu doar crește atractivitatea învățării, ci și eficiența ei.
Proiectul \textit{VisioScience3D} se aliniază acestor direcții moderne de predare,
oferind resurse educaționale inovative care răspund nevoilor noilor generații de elevi.



% Inline Listing example
% \lstset{language=make,caption=Application Makefile,label=lst:app-make}
% \begin{lstlisting}
% CSRCS = app.c
% SRC_DIR =..
% include $(SRC_DIR)/config/application.cfg
% \end{lstlisting}

% \begin{center}
% \begin{table}[htb]
%   \caption{Generated reports - associated Makefile targets and scripts}
%   \begin{tabular}{l*{6}{c}r}
%     Generated report & Makefile target & Script \\
%     \hline
%     Full Test Specification & full_spec & generate_all_spec.py  \\
%     Test Report & test_report & generate_report.py  \\
%     Requirements Coverage & requirements_coverage &
%     generate_requirements_coverage.py   \\
%     API Coverage & api_coverage & generate_api_coverage.py  \\
%   \end{tabular}
%   \label{table:reports}
% \end{table}
% \end{center}


\section{Soluția propusă}
\label{sec:proj}
Ideea platformei este de a crea un mediu de învățare interactiv care să integreze
simulări 3D și tehnologii moderne pentru a face procesul de învățare mult mai ușor.
Oferă o gamă de materii care pot fi studiate în această metodă inovativă, dar poate 
funcționa și ca verifcator ad-hoc al cunoștințelelor elevilor. Un elev poate intra rapid
și facil să verifice o formulă sau altă informație, iar profesorul poate să creeze teste
și să gestioneze clasele de elevi în mod rapid și eficient.

Numele VisioScience3D a fost ales pentru a reflecta scopul platformei și este compus din două
cuvinte: "Visio" care se referă la vizual, vedere, iar "Science" care se referă la știință.
Această combinație sugerează o platformă care îmbină ideea de vizual cu știința, oferind un nume
care reflectă esența platformei și scopul său. Titlul contine și termenul 3D, care subliniază
focusul vizualizărilor din platformă care sunt realizate tridimensional.

Fiecare rol din procesul educațional (elev, profesor) are posibilitatea de accesa funcționalitățile 
de învațăre și evaluare. De asemenea, platforma va avea un sistem de gamificare care va recompensa
elevii pentru progresul lor și va încuraja participarea activă. Opțiuni de vizualizare a tabele
de rezultate vor fi disponibile pentru profesori, iar elevii vor putea să-și urmărească scorul și 
progresul în timp real.

Testele pot fi create prin drag-and-drop în interfața secțiunii de create, unde exist control granular
de la nivel de structura a quizului până la nivel de întrebare, răspuns, selecție de imagini sau număr de 
răspunsuri corecte. Profesorii pot vizualiza clasele pe care le dețin, elevii care au participat la teste
și rezultatele obținute de aceștia. De asemenea, profesorii pot vizualiza și analiza rezultatele elevilor
pentru a înțelege mai bine progresul acestora și pentru a adapta metodele de predare în funcție de nevoile
fiecărui elev. Aceasta va permite o abordare personalizată a învățării, care poate îmbunătăți semnificativ
rezultatele elevilor. Profesorii au access si la sistemul de invitație a elevilor în platformă și în clasă
direct în contul elevului.

Elevii pot accesa platforma printr-o interfață prietenoasă și intuitivă, unde pot explora concepte
complexe prin simulări 3D și animații interactive. Pentru ei este destinat meniul 3D principal de selecție a 
materiei. unde pot vedea și interacționa cu toată gama de simulări disponibile. De asemenea, după cum am menționat
mai sus, elevii pot participa la teste și evaluări pentru a-și verifica cunoștințele. Aceste teste sunt concepute
pentru a fi interactive și captivante, oferind o experiență de învățare plăcută și eficientă, dar fiind și potrivite 
ad-hoc pentru o testare rapidă după o lecție predată.

\section{Rezultate obținute}
\label{sec:proj}

Platforma a ajuns intr-un punct în care poate fi utilizată de către profesori și elevi pentru a explora concepte
și reprezintă o soluție care poate salva timp și poate face învațarea mai rapidă și intuitivă pentru elevii
cu stil de învățare vizual, care după cum am menționat și după cum arată studiile sunt majoritari în școlile
din România (peste 48\% din elevi). La nivelul de profesor reprezintă curent o soluție rapidă de testare știința
monitorizare a elevilor la materii de știința, dar și de informatică.

La nivel tehnic, platforma folosește o arhitectură scalabilă a serviciilor din back-end, oferind o scalabilitate
foarte ridicată și o disponibilitate crescută datorită separării în microservicii a aplicației. Arhitectura poate
fi ușor extinsă pentru a adăuga noi funcționalități și module, iar platforma poate fi adaptată rapid la nevoile
utilizatorilor. De asemenea, partea de front-end este construită folosind tehnologii cu suport extins și
comunități mari, ceea ce asigură o suport îndelungat și o posibilă dezvoltare ușoară a platformei în viitor. 

\section{Structura lucrării}
\label{sec:proj}


Această lucrare este structurată în mai multe capitole, fiecare abordând un aspect diferit al proiectului atât din 
punct de vedere tehnic, cât și din punct de vedere al implementării și utilizării platformei.

Capitolul \ref{chapter:analiza} oferă o analiză detaliată a cerințelor și a nevoilor utilizatorilor, precum și a
profilului utilizatorilor tipici ai platformei. Acesta include o descriere a motivației, a cerințelor funcționale și
a celor nefuncționale, precum și a limitărilor și constrângerilor proiectului.

Al treilea capitol (\ref{chapter:studiuPiata}) analizează piața și competiția din punct de vedere al platformelor educaționale
existente, evidențiind punctele forte și slabe ale acestora și modul în care platforma propusă se diferențiază
de cea dezvoltată în cadrul acestui proiect. 

Capitolul \ref{chapter:tehnologii} detaliază tehnologiile utilizate în dezvoltarea platformei, inclusiv limbajele de
programare, framework-urile și instrumentele utilizate pentru a crea aplicația. Aici se oferă o privire de ansamblu 
asupra dezvoltării fiecarei părți importante a aplicației: back-end, front-end (inclusiv UI/UX), scene 3D și bazele de date.


Capitolul \ref{chapter:implementare} oferă detalii despre implementarea platformei, inclusiv bucăți de cod și exemple De
dezvoltare a codului, precum și detalii despre configurarea și conectivitatea obținută între diferitele componente 
ale aplicației.

Cel de-al șaselea capitol, \ref{chapter:scenariiUtilizare} are ca focus definirea și prezentarea scenariilor de utilizare
ale platformei, inclusiv modul în care utilizatorii pot interacționa cu aplicația și cum pot crea conturi, teste, clase
și cum pot vizualiza scene și experimente 3D. De asemenea, se discută despre modul în care utilizatorii pot accesa și utiliza
funcționalitățile platformei, precum și despre modul în care pot beneficia de gamificare și recompense pentru progresul
lor.

Capitolul \ref{chapter:evaluare} se concentrează pe tipurile de evaluare a platformei, cu accent pe testare și pe modul în
care se poate monitoriza sănătatea platformei în cazul lansării către publicul larg. De asemenea se discută și despre
evaluare performanței platformei pe diferite niveluri.


Ultimul capitol, \ref{chapter:concluzii}, oferă concluzii și perspective asupra viitorului platformei, inclusiv
posibile îmbunătățiri și extinderi ale funcționalităților existente. De asemenea, se discută despre impactul pe care
platforma ar putea să-l aibă asupra educației.
\chapter{Analiza problemei / Motivația proiectului}
\label{chapter:analiza}

\section{Plaja de utilizatori}
\label{sec:proj}

\subsection{Categorii}
\label{sub-sec:proj-scope}

\subsection{Profilul utilizatorului}
\label{sub-sec:proj-user-profile}

\section{Motivația proiectului}
\label{sec:proj-motivation}

\section{Cerințe funcționale}
\label{sec:proj-requirements}

\section{Cerințe nefuncționale}
\label{sec:proj-non-functional-requirements}

\section{Limitări}
\label{sec:proj-limitations}

\chapter{Studiu de piață / Soluții existente}
\label{chapter:studiuPiata}

\section{Alte soluții existente}
\label{sec:proj}
\subsection{Colorado Edu}

Prima soluție analizată este Colorado Edu, o platofrmă care oferă acoperire pe multiple materii după cum se poate observa în captura
următoare.

\fig[width=1\textwidth]{imgs/phet.png}{fig:distributie_persoane}{Captură de ecran de pe soluția Phet Colorado Edu}

Se poate observa designul simplu și intuitiv și suportul relativ extins pe materii. Are și o interactivitate bună la nivel de simulări și alt 
avantaj e că are acces gratuit. Totuși, nu are un sistem de evaluare integrat și nu permite crearea de teste personalizate. De asemenea, nu are
un sistem de management al utilizatorilor, ceea ce face ca utilizarea platformei să fie limitată la simulări ad-hoc, iar ținta principală ca științelor
curiculum este SUA și nu România.

\subsection{invatamate.com}


\fig[width=1\textwidth]{imgs/invatamate.png}{fig:invatamate}{Captură de ecran de pe platforma invatamate.com}

Cea de-a doua soluție analizată este invatamate.com, o platformă care oferă o suită de cursuri, vizualizări și simulări online.
După cum se poate observa în captura de ecran, platforma are un design relativ învechit și nu foarte atractiv, dar
conține multiple resurse utile. Deși numele sugerează că platforma este dedicată matematicii, ea are și simulări de astronomie
și joculețe interactive mai generale.


Problema este iar că nu are un sistem de management al utilizatorilor și nici suport pentru evaluare integrat.
Lipsa acestor funcționalități face ca platforma să nu fie foarte utilă în mediul educațional, iar utilizarea ei să fie limitată 
la joculețe simple și triviale. Alte limitări ale platformei sunt că folosește Flash pentru majoritate jocurilor științelor
chiar integrări cu Kahoot pentru unele, iar resursele sunt de asemenea legate din surse externe în unele cazuri.
De asemeea, experiența nu este cea mai modernă, intuitivă și placută după cum se poate vedea în capturile următoare.

\newpage
\begin{figure}[htb]
    \centering
    \begin{minipage}[b]{0.49\textwidth}
        \centering
        \includegraphics[width=\textwidth]{imgs/invatamatesimulare.png}
        \caption{Simulări de pe platforma invatamate.com}
        \label{fig:invatamatesimulare}
    \end{minipage}
    \hfill
    \begin{minipage}[b]{0.49\textwidth}
        \centering
        \includegraphics[width=\textwidth]{imgs/invatamatejoc.png}
        \caption{Jocuri de pe platforma invatamate.com}
        \label{fig:invatamatesjoc}
    \end{minipage}
\end{figure}


\subsection{Mozaweb.com}

Această soluție are o arhitectură cu produse pentru mai multe roluri (pentru elevi, pentru profesori, pentru școli) și o experiență de utilizare
plăcută și intuitivă. Se poate observa în captura de ecran de mai jos cum arată interfața de întâmpinare a utilizatorului.

\fig[width=0.8\textwidth]{imgs/mozaweb.png}{fig:mozaweb}{Captură de ecran de pe platforma mozaweb.com}

Se poate observa suportul pentru numeroase materii și din zona STEM, dar și pentru alte domenii. De asemenea, platforma are simulări foarte
profesioniste și interactive, dar dezavantajele mari sunt prețurile abonametelor pentru că discutăm de o platformă comercială și nu gratuită.
De asemenea pentru simulări este necesar să se instaleze un plugin extern, reducând astfel accesibilitatea platformei. Se poate observa mai jos
gama de simulări și necesitatea plugin-ului.

\fig[width=1\textwidth]{imgs/mozawebplugin.png}{fig:mozaweb}{Simulări și dovada plugin-ului necesar pentru a le rula}

Ținta aceste platforme este una comercială mai mult decât educațională, iar prețurile sunt relativ mari pentru toate tipurile de utilizatori.


\subsection{iqboard.ro}

Ultima soluție analizată este iqboard.ro, o platformă care oferă o suită de resurse educaționale în limba română, cu multiple moduri 
de lucru, multi touch, bazat pe un sistem de abonamente. Are o interfață interesantă și atractivă și suport extins educațional după cum se poate observa
în captura de ecran de mai jos (figura \ref{fig:iqboard}). 

Dezavantajele majore sunt prețurile extrem de mari pentru utilizatori (la nivel de sute de euro după cum se poate observa în figura \ref{fig:iqboard}).
Prețul este relativ nejustificat deși materialele sunt de calitate bună și platforma are un design interactiv și atractiv.
în figura \ref{fig:iqboardsimulari} se poate urmări un exemplu de simulare disponibilă pe platformă. Platforma are moduri separate pentru 
pregătire, predare și mod desktop, dar nu are un sistem de management al utilizatorilor sau vreun sistem de evaluare sau 
urmărire a progresului utilizatorilor.

Ca alte funcționalități, platforma se laudă cu modul multi-user care poate fi folosit in modul tabla sau full-screen.
Dupa selectarea instrumentului dorit, utilizatorii pot lucra simultan pe tabla, cu modul multi-screens (modul petrecere), 
cu posibilitate de folosire simultana a doua table interactive (modul MX2) și cu modul raspuns interactiv.

\begin{figure}[htb]
    \centering
    \begin{minipage}[b]{0.49\textwidth}
        \centering
        \includegraphics[width=\textwidth]{imgs/iqboard.png}
        \caption{Captură de ecran de pe platforma iqboard.ro}
        \label{fig:iqboard}
    \end{minipage}
    \hfill
    \begin{minipage}[b]{0.49\textwidth}
        \centering
        \includegraphics[width=\textwidth]{imgs/iqboardsimulari.jpg}
        \caption{Simulări de pe platforma iqboard.ro}
        \label{fig:iqboardsimulari}
    \end{minipage}
\end{figure}


\section{Raportarea la alte soluții}
\label{sub-sec:proj-scope}


\begin{itemize}
    \item \textbf{O singură platformă completă pentru toate materiile STEM}
    În timp ce Phet sau Invatamate se concentrează mai mult pe un subiect, VisioScience3D
    va oferi un mediu unificat cu module interactive de matematică, fizică, chimie, informatică și
    chiar astronomie, reducând necesitatea schimbării constante de aplicații.
    \item \textbf{Simulări 3D real-time și configurabile}
    Mozaweb de exemplu are mult suport pentru scene 3D, dar ele sunt predefinite şi nu pot fi adaptate dinamic. 
    VisioScience3D permite ajustarea parametrilor în timp real (forțe, unghiuri, constante) și animarea vectorilor.
    \item \textbf{Dashboard de monitorizare a progresului}
    Lipsa de raportare în timp real este o problemă la majoritatea soluţiilor gratuite. 
    Profesorii pot vizualiza instantaneu statistici legate de scoruri, dificultăţi întâmpinate și 
    pot adapta testele în funcție de nevoile elevilor.
    \item \textbf{Gamificare}
    Majoritatea platformelor nu au un sistem de gamificare bine definit. VisioScience3D va include elemente
    de gamificare pentru a spori implicarea elevilor.
\end{itemize}

\section{Motivația alegerii VisioScience3D de către utilizatori}
\label{sub-sec:proj-motivatie}

VisioScience3D răspunde foarte bine nevoilor profesorilor și elevilor de azi.
Oferă simultan simulări 3D real-time pentru toate disciplinele STEM, configurabile 
direct în browser în platformă, fără pluginuri externe. În timp ce multe platforme 
existente fie acoperă doar un număr limitat de subiecte, fie implică licențe costisitoare
sau dependențe externe (Mozaweb, IQBoard). VisioScience3D pune la dispoziție un mediu unic,
adaptat programei românești. Profesorii pot monitoriza progresul elevilor prin dashboard-uri
integrate și primesc rapoarte detaliate. Prin modelul gratis și nivelul simulărilor suportate,
VisioScience3D este un ecosistem complet, flexibil și orientat spre rezultatele elevilor.
\chapter{Tehnologii utilizate pentru front-end}
\label{chapter:tehnologii}

\label{sec:proj}
\section{Soluția de front-end}
\subsection{Framework principal}

Frameworkul principal pe care l-am folosit este React JS, care este o tehnologie foarte populară
și răspândită bazată pe JavaScript. În alegerii de tehnologie pentru front-endul platformei VisioScience3D,
am concentrat atenția pe criterii precum maturitatea ecosistemului, performanță, curba de învățare,
flexibilitate în personalizare şi capacitatea de integrare cu biblioteci 3D. Dintre principalele
opțiuni (Vue.js, Angular, Svelte), React s-a impus datorită următoarelor avantaje:

\begin{itemize}
\item
  \textbf{Ecosistem matur și susținere largă}:
    React are o comunitate activă de milioane de dezvoltatori,
    cu un număr impresionant de librării, tutoriale și un suport consistent din partea Meta. 
    Această resursă ne-a permis să adoptăm rapid bune practici și să găsim soluții la provocări 
    specifice dezvoltării 3D.
\item
  \textbf{Model declarativ și component-driven}:
    React oferă un mod de declarare detaliată în descrierea interfeței, ideal pentru scene 3D
    complexe, unde starea se propagă predictibil prin componente. Aceasta ușurează managementul 
    ciclului de viață al obiectelor din biblioteca 3D, reducând riscul de actualizări inconsistente.

\item
  \textbf{React Router DOM}:
    Pentru că oferă o soluție de rutare foarte ușoara în navigarea între diferitele
    secţiuni ale platformei (fizică, chimie, astronomie etc.) prin React Router DOM 
    ce oferă simplitate în definirea rutelor şi suportul pentru incărcare întârziată de module.
    Alternativele (Vue Router, Angular Router) sunt la fel de capabile, însă integrarea lor cu
    React Three Fiber, care vom vedea este baza în simulările 3D, ar fi impus un strat suplimentar
    de interoperabilitate.
\end{itemize}


Pe zona de stilizare a interfeței, am optat pentru Tailwind CSS, un framework CSS utilitar 
care ne-a permis să creăm un design personalizat. Am comparat framework-uri clasice de CSS
(Bootstrap, Material UI) cu Tailwind CSS și am observat următoarele avantaje:
\begin{itemize}
\item
  \textbf{Utilitare-întâi și JIT}:
    Tailwind oferă clase utilitare care permit definirea rapidă a
    layout-urilor şi a stilurilor unice, fără a scrie sutele de linii de CSS personalizat. Motorul
    JIT (Just-In-Time) generează doar clasele folosite, menținând pachetul final mai mic.
\item
  \textbf{Consistență și flexibilitate}:
    În loc să ne încărcăm aplicația cu componente predefinite,
    am putut construi ună interfață coerentă, respectând paleta de culori şi spațierile dorite,
    fără a rescrie componente UI complexe.
\item
  \textbf{Integrare strânsă cu React}:
    Utilizarea className din JSX se potrivește natural cu Tailwind,
    iar plugin-urile precum @tailwindcss/forms facilitează stilizarea
    formularelor şi input-urilor.
\item
  \textbf{Extensibilitate}:
    Tailwind permite personalizarea ușoară a temelor, adăugarea de plugin-uri și crearea de 
    componente reutilizabile, fără a sacrifica viteza de dezvoltare.
\end{itemize}

Bibliotecile de creare și gestionare a graficii 3D sunt esențiale pentru platforma noastră,
iar pentru a crea simulări interactive, am ales Three.js, o bibliotecă JavaScript populară.
Three.js oferă un API puternic pentru crearea de scene 3D, animații și interacțiuni și Împreună
cu React Three Fiber (R3F), o bibliotecă care integrează Three.js cu React, am putut să ne concentrăm
pe dezvoltarea rapidă a aplicațiilor 3D fără a ne preocupa de detaliile de implementare ale Three.js.
Utilizarea peste a Drei, ce conține o suită de componente și utilitare pentru R3F, a permis accelerarea
procesului de dezvoltare, oferind soluții gata făcute pentru anumite probleme comune.

Câteva dintre beneficiile integrării cu Three.js prin React Three Fiber și Drei care au făcut 
alegerea justificată sunt:

\begin{itemize}
\item
  \textbf{Abstracție declarativă}:
    R3F “împachetează” API-ul imperativ Three.js într-un DSL React, 
    permițându-ne să definim scene, obiecte și materiale în JSX, sub formă de componente. 
    Astfel, putem folosi hook-uri (useFrame, useLoader) pentru a sincroniza animațiile și 
    resursele fără prea mult cod duplicat.
\item
  \textbf{Ecosistem Drei}:
    Biblioteca Drei aduce peste 100 de componente existente (OrbitControls,
    Text, etc.), accelerate de comunitate, care au economisit timp de
    implementare.
\item
  \textbf{Optimizări de performanță}:
    Datorită reconciler-ului React, R3F actualizează doar părțile
    din scenă care se schimbă.
\end{itemize}

Așadar, combinația React + React Router DOM + Tailwind CSS + React Three Fiber + Drei ne-a oferit un
stack unificat, performant și ușor de întreținut, bine adaptat
pentru dezvoltarea rapidă de simulări 3D interactive în browser și pentru extinderea ușoară a
platformei VisioScience3D. Se poate observa în figura \ref{fig:fediagram} cum arată
arhitectura simplificată a frontend-ului platformei.

\fig[width=1.1\textwidth]{imgs/fediagram.png}{fig:fediagram}{Diagrama de design principală a frontend-ului}

\subsection{Framework 3D}

După cum am discutat anterior, frameworkul 3D ales este React Three Fiber, care
este o bibliotecă care integrează Three.js cu React, permițându-ne să creăm scene 3D
într-un mod declarativ și reactiv. Aceasta se bazează pe principii de WebGL combinate
cu ce poate oferi React, cum ar fi componentizarea și gestionarea stării. R3F ne permite să
creăm scene 3D complexe folosind JSX și să integrăm hook-uri ca:

\begin{itemize}
\item
  \textbf{useFrame}:
    Acesta permite să actualizăm scena la fiecare cadru, facilitând animațiile și interacțiunile
    în timp real.
\item
  \textbf{useLoader}:
    Acesta ajută să încărcăm resurse externe (texturi, modele 3D) într-un mod eficient,
    folosind promisiuni pentru a gestiona încărcarea asincronă.
\item
  \textbf{useThree}:
    Acesta oferă acces la obiectul Three.js curent, permițându-ne să interacționăm direct
    cu scena, camera și renderer-ul.
\item
  \textbf{useRef}:
    Acesta permite să creăm referințe la obiectele din scenă, facilitând manipularea lor
    direct în timpul animațiilor sau interacțiunilor.
\item
  \textbf{useState}:
    Acesta permite să gestionăm starea componentelor, facilitând actualizarea și redarea
    dinamică a obiectelor din scenă.
\item
  \textbf{useEffect}:
    Acesta permite să gestionăm efectele secundare, cum ar fi adăugarea de evenimente sau
    actualizarea stării în funcție de modificările din scenă.
\end{itemize}

Pentru integrarea obiectelor cu extensia .glb am folosit un convertor gltf  și anume [20] care
ne permite să convertim modele .glb în componente React Three Fiber oferind grupuri de meshe-uri și
materiale. Acest lucru a permis să importăm modelele 3D direct în aplicația React și să creăm
diverse manipulări și animații folosind API-ul R3F.


\fig[width=1\textwidth]{imgs/convertor.png}{fig:fediagram}{Exemplu de convertire a unui model 3D .glb în R3F}


\subsection{Modelare 3D}
Am folosit Blender pentru a crea modelele 3D utilizând modele și primitive existente pe internet. De exemplu
modelul paginii principale a fost făcut din primitive de insule și castele și adaptat la un număr de 
insule potrivit materiilor din meniu, cât și loc pentru suport viitor pentru alte materii. 

Putem observa în figura următoare dezvoltarea modelului în Blender.

\fig[width=1\textwidth]{imgs/Blender.png}{fig:fediagram}{Modelul paginii principale în Blender}

La acest model s-au adăugat animații de rotație și schimbare a state-ului la mișcarea mouse-ului,
animație continuă de plutire și animații realiste de zbor pentru dronele din jurul insulelor.


\section{Soluția UI/UX}
\label{sub-sec:proj-ui-ux}
\subsection{Brandingul proiectului}
\subsubsection{Paleta de culori}
Brandingul VisioScience3D se bazează pe o paletă caldă, prietenoasă şi destul de pastelată,
menită să creeze o atmosferă de joc și luminoasă pentru elevi. Fundalurile folosesc un gradient
subtil de la \#fdf4ff (alb-roz pal) prin \#f3e8ff (lavandă delicată) spre \#fff7ed (piersică pal),
în timp ce culorile de accent dau personalitate interfeţei: nuanţa profundă de mulberry (\#690375)
marchează elementele active, bordurile şi textele importante, iar tonul rosy-brown (\#AE847E) 
individualizează secţiunile de chimie şi astronomie. Pentru evidenţierea vizualizărilor 3D
am introdus un violet închis (\#4f46e5). Culorile reflectă o atmosferă dinamică și 
energică, încurajând explorarea și învățarea.

\fig[width=1\textwidth]{imgs/paleta.png}{fig:fediagram}{Paleta de culori a platformei}

\subsubsection{Logo-ul}
Logo-ul VisioScience3D este relativ simplu și în temă cu paleta de culori, fiind reprezentat
de cuvantul "VisioScience3D" scris cu un font prezentativ și colorat cu un gradient de la
mov la mulberry (\#690375).

\fig[width=0.5\textwidth]{imgs/logo.png}{fig:fediagram}{Logo-ul platformei}


\subsubsection{Mascota platformei}
Mascota proiectului este un balon care se apropie de paleta de culori a platformei
și care simbolizează explorarea și învățarea. Balonul apare în toate paginile importante ale platformei
și este de fiecare dată animat potrivit acțiunilor paginii respective. De exemplu, pe pagina de 
înregistrare mascota se mișcă în sus și în jos când utilizatorul scrie în câmpurile de
înregistrare, iar pe la înregistrare reușită mascota sare în sus și în jos pentru câteva secunde.
Balonul mascotă este și principalul caracter de explorare a meniului principal, învârtindu-se între
insule pentru a ajunge la diferitele materii.

\fig[width=0.35\textwidth]{imgs/mascota.png}{fig:fediagram}{Mascota platformei}

Se poate observa și o captură în timpul animației facută la succesul la înregistrare.

\fig[width=0.6\textwidth]{imgs/animatieregister.png}{fig:fediagram}{Captură de ecran a mascotei la înregistrare}

\subsection{Experiența utilizatorului}
\subsubsection{Navigarea în platformă}
\subsubsection{Crearea componentelor UI}
\subsubsection{Designul interfeței utilizatorului}

\section{Soluția de creare a scenelor 3D}
\label{sec:proj-3d}
\subsection{Utilizarea tehnologiilor WebGL și Three.js}
\subsubsection{Crearea și gestionarea scenelor 3D}
\subsubsection{Interacțiunea cu obiectele 3D}
\subsubsection{Optimizarea performanței graficii 3D}
\subsection{Integrarea cu frontendul}








\chapter{Tehnologii utilizate pentru back-end}
\label{sec:proj-backend}
\section{Descrierea arhitecturii}
\subsection{Configurarea clusterului}
\subsection{Rutarea și gestionarea traficului}

\section{Descrierea serviciilor}
\subsection{Descrierea API-urilor}

\subsection{Baze de date}

\section{Securitate}

\section{CI/CD}
\subsection{Deployment}
\subsection{Metrici / Monitorizare}


\section{Soluția de baze de date}
\label{sec:proj-database}
\subsection{Structura bazei de date}
\subsubsection{Tipuri de baze de date utilizate / Motivația alegerii}
\subsubsection{Gestionarea datelor}
\subsection{Securitatea datelor}
\subsubsection{Backup și restaurare}
\subsubsection{Performanța și scalabilitatea bazei de date}
\subsection{Integrarea cu back-endul}
\chapter{Detalii de implementare}
\label{chapter:implementare}

\section{Configurare back-end}
\label{sec:implementare-backend}

\section{Dezvoltare back-end}
\label{sec:implementare-backend}

\subsection{Configurare front-end}
\label{subsec:implementare-frontend}

\subsection{Dezvoltare front-end}
\label{subsec:implementare-frontend}
\include{src/chapters/funcionalitati}
\chapter{Scenarii de utilizare}
\label{chapter:implementare}

\section{Înregistrare si autentificare utilizator}
\label{sec:inregistrare-utilizator}

\section{Explorarea meniul principal}
\label{sec:explorare-meniu-principal}

\section{Accesarea secțiunilor educaționale}
\label{subsec:accesare-sectiuni-educationale}

\section{Interacțiunea cu scenele 3D educaționale}
\label{subsec:interactiune-scene-3d}

\section{Gestionarea profilului de profesor}
\label{sec:gestionare-profil-utilizator}

\subsection{Crearea de clase și gestionarea elevilor}
\label{subsec:creare-clase-elevi}

\subsection{Crearea de teste și gestionarea lor}
\label{subsec:creare-teste}

\subsection{Vizualizarea rezultatelor}
\label{subsec:vizualizare-rezultate}

\section{Gestionarea contului de elev}
\label{subsec:gestionare-cont-elev}

\subsection{Intrarea în clasele profesorului}
\label{subsec:intrare-clase-profesor}

\subsection{Accesarea testelor și vizualizarea rezultatelor}
\label{subsec:accesare-teste-elev}

\subsection{Rezolvarea testelor}
\label{subsec:rezolvare-teste}
\chapter{Evaluarea implementării}
\label{chapter:evaluare}

\section{Evaluarea back-endului}
\label{subsec:evaluare-backend}

\subsection{Testarea back-endului}
\label{subsubsec:evaluare-backend-testare}

\subsection{Monitorizarea back-endului}
\label{subsubsec:evaluare-backend-monitorizare}

\subsection{Evaluarea performanței back-endului}
\label{subsubsec:evaluare-backend-performanta}

\section{Evaluarea front-endului}
\label{subsec:evaluare-frontend}

\subsection{Testarea front-endului}
\label{subsubsec:evaluare-frontend-testare}

\subsection{Monitorizarea front-endului}
\label{subsubsec:evaluare-frontend-monitorizare}

\subsection{Evaluarea performanței front-endului}
\label{subsubsec:evaluare-frontend-performanta}

\section{Testarea infrastructurii / Platformei}
\label{subsec:evaluare-infrastructura}


\chapter{Concluzii și perspective}
\label{chapter:concluzii}

\section{Concluzii}
\label{sec:concluzii}

\section{Dezvoltare viitoare}
\label{sec:dezvoltare-viitoare}
% \chapter{Bibliografie}
\label{chapter:bibliografie}

\section{Referințe bibliografice}
\label{sec:bibliografie-referinte}


% \chapter{Anexe}
\label{chapter:anexe}


% \include{src/chapters/second-chapter}



\appendix
\input{src/appendix/index}

% \bibliography{src/main}
% \bibliographystyle{plain}

\begin{thebibliography}{99}

\bibitem{threejs}
\textit{Three.js}. \href{https://threejs.org/}{https://threejs.org/}.

\bibitem{webgl}
\textit{WebGL}. \href{https://www.khronos.org/webgl/}{https://www.khronos.org/webgl/}.

\bibitem{react}
\textit{React}. \href{https://reactjs.org/}{https://reactjs.org/}.

\bibitem{nodejs}
\textit{Node.js}. \href{https://nodejs.org/}{https://nodejs.org/}.

\bibitem{mongodb}
\textit{MongoDB}. \href{https://www.mongodb.com/}{https://www.mongodb.com/}.

\bibitem{expressjs}
\textit{Express.js}. \href{https://expressjs.com/}{https://expressjs.com/}.

\bibitem{docker}
\textit{Docker}. \href{https://www.docker.com/}{https://www.docker.com/}.

\bibitem{kubernetes}
\textit{Kubernetes}. \href{https://kubernetes.io/}{https://kubernetes.io/}.

\bibitem{git}
\textit{Git}. \href{https://git-scm.com/}{https://git-scm.com/}.

\bibitem{github}
\textit{GitHub}. \href{https://github.com/}{https://github.com/}.

\bibitem{ci-cd}
\textit{Continuous Integration and Continuous Deployment}. \href{https://www.atlassian.com/continuous-delivery/ci-vs-ci-vs-cd}{https://www.atlassian.com/continuous-delivery/ci-vs-ci-vs-cd}.

\bibitem{go}
\textit{Go Language}. \href{https://golang.org/}{https://golang.org/}.

% === Studii și articole generale ===

\bibitem{study1}
M. Horáková, L. Kovářová și P. Doležel, “3D Models and Animations in STEM Education: Czech Experiment,” *Central European Journal of Education*, 2019. \href{https://link.springer.com/article/10.1007/s10639-024-13210-z}{https://link.springer.com/article/10.1007/s10639-024-13210-z}.

\bibitem{study2}
M. Ionescu și A. Popescu, “Distribuția stilurilor de învățare în rândul elevilor din România,” *Revista Profesorului*, 2020. \href{https://revistaprofesorului.ro/studiu-privind-invatarea-vizuala/}{https://revistaprofesorului.ro/studiu-privind-invatarea-vizuala/}.

\bibitem{study3}
A. Miller, “Learning Styles Among School Students: A Pilot Study,” *British Journal of Educational Psychology*, 2001. \href{https://iteach.ro/pagina/1113/}{https://iteach.ro/pagina/1113/}.

\bibitem{study4}
Y. Zhang et al., “VR/AR in STEM Learning,” *PubMed*, 2024. \href{https://pubmed.ncbi.nlm.nih.gov/39806348/}{https://pubmed.ncbi.nlm.nih.gov/39806348/}.

\bibitem{study5}
Frontiers in Education, “Gamification and Immersive Learning Environments,” 2024. \href{https://www.frontiersin.org/journals/education/articles/10.3389/feduc.2024.1354526/full}{https://www.frontiersin.org/journals/education/articles/10.3389/feduc.2024.1354526/full}.

\bibitem{study6}
Mindomo, “What is Visual Learning?,” 2023. \href{https://www.mindomo.com/blog/what-is-visual-learning/}{https://www.mindomo.com/blog/what-is-visual-learning/}.

\bibitem{study7}
OECD Education Today, “Can the Targeted Use of Digital Devices Improve Learning?,” 2024. \href{https://oecdedutoday.com/can-the-targeted-use-of-digital-devices-in-education-win-over-the-naysayers/}{https://oecdedutoday.com/can-the-targeted-use-of-digital-devices-in-education-win-over-the-naysayers/}.

\end{thebibliography}


\printindex

\end{document}